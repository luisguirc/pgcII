% ---
% RESUMOS
% ---

% RESUMO em português
\setlength{\absparsep}{18pt} % ajusta o espaçamento dos parágrafos do resumo
\begin{resumo}

O presente trabalho realiza uma Revisão Sistemática da Literatura (RSL) com o objetivo de identificar critérios, recomendações e metodologias aplicadas no desenvolvimento de jogos sérios voltados à população idosa, com foco especial em cibersegurança. O envelhecimento populacional e a crescente inserção digital dos idosos tornam esse público mais vulnerável a ameaças online, reforçando a importância de soluções educacionais acessíveis e eficazes. A pesquisa analisou 62 estudos relevantes, destacando lacunas no uso de diretrizes específicas de segurança digital nos jogos existentes. Além disso, observou-se uma carência de abordagens participativas no levantamento de requisitos. Como contribuição, o estudo propõe a consolidação de recomendações que sirvam como guia prático para desenvolvedores e pesquisadores interessados em promover a inclusão digital segura da população idosa por meio de jogos sérios.


 \textbf{Palavras-chaves}: Jogos Sérios, Cibersegurança, Pessoas Idosas, Design Centrado no Usuário, Inclusão Digital.
\end{resumo}

% ABSTRACT in english
\begin{resumo}[Abstract]
 \begin{otherlanguage*}{english}
    This work presents a Systematic Literature Review (SLR) aiming to identify criteria, requirements, and methodologies applied in the development of serious games for the elderly population, with a special focus on cybersecurity. The aging of the population and the increasing digital inclusion of older adults make this group more vulnerable to online threats, highlighting the need for accessible and effective educational solutions. The review analyzed 62 relevant studies, revealing gaps in the application of specific digital security guidelines in existing games. Furthermore, a lack of participatory approaches in requirements elicitation was identified. As a contribution, the study proposes a set of consolidated recommendations to serve as a practical guide for developers and researchers interested in promoting safe digital inclusion for the elderly through serious games.

   \vspace{\onelineskip}
 
   \noindent 
   \textbf{Keywords}: Serious Games, Cybersecurity, Older Adults, User-Centered Design, Digital Inclusion.
 \end{otherlanguage*}
\end{resumo}