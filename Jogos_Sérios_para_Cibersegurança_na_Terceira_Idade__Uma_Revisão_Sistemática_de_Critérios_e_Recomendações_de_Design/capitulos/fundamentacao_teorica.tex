\chapter{Fundamentação Teórica}
\label{cap:fundamentacao_teorica}

A seguir, será apresentada a definição e o escopo dos jogos sérios, além de suas distinções com sistemas gamificados, essenciais para entender o impacto desses jogos no contexto da saúde e bem-estar dos idosos. Outros conceitos como o Design Centrado no Usuário também são introduzidos.

\section{Jogos Sérios}

Jogos sérios são aplicações digitais que combinam elementos de entretenimento e multimídia com o propósito explícito de promover experiências voltadas para objetivos além do divertimento. Esses jogos engajam os usuários por meio de mecânicas lúdicas, mas diferenciam-se dos jogos puramente recreativos por incorporarem conteúdos educacionais, instrucionais, terapêuticos ou de treinamento. A definição mais amplamente aceita caracteriza jogos sérios como aqueles cujo propósito principal não é o entretenimento, ainda que este seja utilizado como meio para facilitar a assimilação de conhecimentos, habilidades ou valores. Assim, o termo “sério” refere-se à intenção de transmitir ao jogador algum tipo de conteúdo significativo, oriundo de experiências ou saberes aplicáveis a contextos reais como saúde, educação ou comunicação interpessoal~\cite{laamarti2014overview}.

A literatura destaca que os jogos sérios são caracterizados por três componentes fundamentais: experiência, entretenimento e multimídia~\cite{laamarti2014overview}. A experiência diz respeito à vivência do jogador com o conteúdo proposto, o entretenimento atua como fator de engajamento e motivação, e os recursos multimídia abrangem os canais sensoriais utilizados na interação (visuais, auditivos, táteis etc.). Diferentes perspectivas acadêmicas e industriais concordam que o sucesso desses jogos reside na capacidade de equilibrar o conteúdo sério com uma experiência envolvente e interativa~\cite{nazry2017mood}. Dessa forma, jogos sérios constituem uma ferramenta promissora para impactar positivamente diversas áreas da sociedade, desde a educação formal até o treinamento militar e a reabilitação médica~\cite{michael2005serious, canapa2025interactive}.

\section{Gamificação vs Jogos Sérios}

É importante distinguir jogos sérios de sistemas gamificados. A \textit{gamificação} refere-se à aplicação de elementos de jogos (como pontos, rankings e recompensas) em contextos não lúdicos, como aplicativos de saúde ou educação \cite{deterding2011gamefulness}. Já os \textit{jogos sérios} são sistemas completos, com mecânicas próprias e objetivos definidos. Para idosos, os jogos sérios oferecem experiências mais imersivas e significativas do que a simples gamificação de tarefas.

\section{Design Centrado no Usuário Idoso}

O desenvolvimento de jogos para idosos exige atenção especial às diretrizes de design centrado no usuário \cite{abras2004user}, priorizando \textit{usabilidade} e \textit{acessibilidade}. Barreiras comuns enfrentadas por idosos incluem visão reduzida, dificuldades auditivas, lentidão na resposta motora e declínio cognitivo. Portanto, de acordo com o W3C \cite{w3c2024wcag}, interfaces devem ser simples, apresentar textos legíveis, ter feedbacks claros e tempos de resposta ajustáveis.

Segundo \cite{fisk2009designing}, é essencial considerar fatores como familiaridade com tecnologia, ergonomia do dispositivo e linguagem acessível. Outro ponto importante é oferecer autonomia e controle ao usuário, respeitando seu ritmo e suas limitações.

\section{Aspectos do Envelhecimento e Interação Digital}

Com o envelhecimento, é comum a ocorrência de declínio cognitivo (como redução da memória de curto prazo e atenção), bem como diminuição da acuidade visual e coordenação motora \cite{charness2009aging}. Estudos mostram que jogos digitais podem atenuar esses efeitos ao estimular áreas específicas do cérebro e manter os usuários engajados cognitivamente \cite{green2008plasticity}.

Além disso, jogos podem servir como ferramentas de socialização, ajudando a combater o isolamento social — um fator de risco para doenças mentais e declínio funcional em idosos \cite{cotten2012internet}.

\section{Tecnologias Aplicadas em Jogos para Idosos}

Diversas tecnologias têm sido utilizadas no desenvolvimento de jogos sérios voltados para o público idoso. Tablets e smartphones com interfaces sensíveis ao toque são preferidos por sua simplicidade de uso \cite{sayago2011everyday}. Tecnologias como reconhecimento de voz, sensores de movimento e realidade aumentada também são promissoras, pois oferecem formas naturais de interação \cite{gerling2012game}. No entanto, o uso dessas tecnologias exige cuidados com acessibilidade e custo, além do esforço necessário para ajustar os equipamentos ou preparar o ambiente em que o jogo será executado.
