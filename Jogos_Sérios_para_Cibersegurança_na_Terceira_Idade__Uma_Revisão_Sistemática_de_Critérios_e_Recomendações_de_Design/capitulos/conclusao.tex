\chapter{Conclusões}
\label{cap:conclusoes}

\section{Contribuições}

Este trabalho apresenta contribuições relevantes para o campo do desenvolvimento de jogos sérios voltados à cibersegurança para pessoas idosas. A principal contribuição consiste na realização de uma RSL que identifica e organiza critérios, recomendações e abordagens metodológicas aplicadas no design de jogos voltados para esse público, com ênfase em segurança digital. O estudo oferece uma visão abrangente do estado da arte, destacando lacunas importantes na literatura, especialmente no que se refere à ausência de diretrizes consolidadas e práticas participativas de levantamento de diretrizes sensíveis às necessidades cognitivas, físicas e tecnológicas dos idosos.

Além disso, o trabalho propõe um conjunto de recomendações que pode ser utilizado como ponto de partida para desenvolvedores e pesquisadores interessados em criar soluções mais acessíveis, inclusivas e eficazes. A sistematização de boas práticas encontradas nos estudos analisados contribui não apenas para a melhoria da qualidade dos jogos desenvolvidos, mas também para a promoção da inclusão digital segura desse grupo social cada vez mais conectado, porém vulnerável a ameaças cibernéticas.

Ao delimitar claramente os critérios de avaliação e exclusão de estudos, este trabalho também contribui metodologicamente para futuras revisões e pesquisas similares, fortalecendo a base teórica e empírica sobre o tema.

\section{Limitações}

Durante a etapa de busca dos artigos, observou-se uma predominância de estudos envolvendo jogos digitais que utilizam sensores de movimento, com destaque para os chamados \textit{exergames} — jogos desenvolvidos com o objetivo de promover a prática de atividades físicas. Considerando que esta revisão sistemática tem como foco a identificação de recomendações gerais para o desenvolvimento de jogos sérios que não dependam de aparatos tecnológicos específicos, como sensores externos, optou-se pela exclusão desses estudos por meio do critério de exclusão 6.

Além disso, constatou-se que muitos trabalhos não tinham como público-alvo principal a população idosa, mas sim indivíduos acometidos por condições clínicas frequentemente associadas ao envelhecimento. Como esta revisão busca compreender as necessidades específicas de pessoas idosas no contexto do desenvolvimento de jogos digitais, foram desconsiderados os estudos centrados em patologias específicas que não destacavam de forma clara a população idosa.

\section{Trabalhos Futuros}

Este estudo apresentou etapas iniciais da execução de uma Revisão Sistemática da Literatura (RSL) sobre o tema proposto. Nas próximas fases do Projeto de Graduação em Computação, conforme descrito na Seção~\ref{sec:cronograma}, está prevista a continuidade da RSL, incluindo a análise e discussão de novos resultados.

Além disso, propõe-se como trabalho futuro o desenvolvimento de um protótipo de jogo sério que incorpore as recomendações identificadas nesta revisão. A validação do protótipo deverá ser conduzida com a participação do público-alvo, bem como de cuidadores, especialistas em design de jogos e profissionais da área de cibersegurança, a fim de avaliar sua usabilidade, eficácia pedagógica e adequação às necessidades específicas dos usuários.
