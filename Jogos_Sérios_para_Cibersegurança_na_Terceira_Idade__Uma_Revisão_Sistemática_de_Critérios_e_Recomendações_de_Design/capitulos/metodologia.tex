\chapter{Metodologia}
\label{cap:metodologia}

Para identificar os critérios e recomendações fundamentais para o desenvolvimento de jogos sérios sobre cibersegurança para pessoas idosas, foi adotada a metodologia de Revisão Sistemática da Literatura (RSL). Esta abordagem permite uma investigação científica rigorosa e reprodutível da literatura existente, seguindo um protocolo bem definido, conforme descrito por \cite{kitchenham2004procedures}.

\section{Protocolo da RSL}\label{sec:protocolo_rsl}

\subsection{Questões de Pesquisa}\label{subsec:questoes_pesquisa}

As seguintes questões de pesquisa foram definidas para nortear a RSL:
\begin{enumerate}
    \item Quais as recomendações, requisitos, e critérios identificados no desenvolvimento de jogos para o público idoso?
    \item Como jogos desenvolvidos tendo o público idoso como alvo são validados e testados?
    \item Que abordagens são usadas para levantamento de requisitos nestes jogos e quais produzem melhores resultados?
\end{enumerate}

\subsection{Critérios de Inclusão e Exclusão}\label{subsec:criterios}

Para garantir a relevância e qualidade dos estudos selecionados, foram estabelecidos os seguintes critérios:

\textbf{Critérios de inclusão:}
\begin{enumerate}
    \item Apresenta recomendações específicas voltados ao público idoso em contextos de jogos ou tecnologias interativas;
    \item Discute metodologias de levantamento de requisitos com participação do público idoso ou voltadas a suas características;
    \item Inclui avaliação empírica ou reflexões sobre a efetividade dos critérios adotados para o público idoso;
    \item Relaciona as recomendações e critérios com aspectos de cibersegurança ou proteção digital.
\end{enumerate}

\textbf{Critérios de exclusão:}
\begin{enumerate}
    \item Trabalhos que não sejam artigos de pesquisa;
    \item Trabalhos que não estejam escritos em Língua Portuguesa ou Inglesa;
    \item Trabalhos duplicados ou que descrevam o mesmo projeto;
    \item Trabalhos publicados antes de 2015;
    \item Trabalhos que mencionam o público idoso apenas de forma tangencial, sem foco na proposta de design ou avaliação de jogos para esse público;
    \item Trabalhos que fazem uso de sensores de movimento como principal característica;
    \item Revisões de literatura;
    \item Trabalhos não disponíveis para acesso;
\end{enumerate}

\subsection{Bases de Dados e Estratégia de Busca}\label{subsec:bases_busca}

A busca foi realizada nas seguintes bases de dados:
\begin{itemize}
    \item IEEE Xplore
    \item ACM Digital Library
    \item Web of Science
    \item SBC OpenLib (para publicações nacionais)
\end{itemize}

As bases de dados foram escolhidas por sua ampla relevância e abrangência na área de Ciência da Computação, Engenharia e Tecnologia da Informação. A escolha dessas bases visa garantir a abrangência, diversidade e confiabilidade dos estudos coletados e são amplamente utilizadas em revisões sistemáticas e pesquisas científicas. Para cada base, foram utilizadas strings de busca adaptadas à sua sintaxe específica, mas mantendo a mesma lógica de combinação de termos relacionados a:

\begin{enumerate}
    \item Jogos: \textit{game, games, gaming, jogo, jogos}
    \item Educacional/Sério: \textit{educational, serious, educacional, sério}
    \item Público-alvo: \textit{elderly, old adult, old adults, senior, idoso, terceira idade}
\end{enumerate}

Estes termos foram buscados no título, abstract, e em palavras-chave nas bases que permitiam tal busca. Por exemplo, a string utilizada na IEEE Xplore foi:
\begin{verbatim}
("All Metadata":"gaming" OR "All Metadata":"game" OR 
 "All Metadata":"games") AND 
("All Metadata":"educational" OR "All Metadata":"serious") AND 
("All Metadata":"elderly" OR "All Metadata":"old adult" OR 
 "All Metadata":"old adults" OR "All Metadata":"senior")
\end{verbatim}

\subsection{Processo de Seleção}\label{subsec:processo_selecao}

O processo de seleção dos artigos foi realizado em quatro fases:
\begin{enumerate}
    \item \textbf{Fase 1}: Todos os artigos retornados pela pesquisa
    \item \textbf{Fase 2}: Remoção de artigos duplicados, escritos em outras línguas que não Inglês ou Português, ou outros tipos de trabalhos (definido pelo critério de exclusão 1)
    \item \textbf{Fase 3}: Leitura do título e \textit{abstract} e aplicação dos critérios de inclusão e exclusão
    \item \textbf{Fase 4}: Leitura do texto completo dos artigos, aplicação dos critérios de inclusão e exclusão (feita), e extração e análise dos dados relevantes (a ser realizada nas próximas etapas do PGC).
\end{enumerate}

Os detalhes do número de artigos encontrados e selecionados em cada fase são apresentados no Capítulo~\ref{cap:resultados}.

\section{Cronograma de Trabalho}\label{sec:cronograma}

O desenvolvimento deste projeto está organizado em três fases, correspondentes aos três quadrimestres do Projeto de Graduação em Computação (PGC-I, PGC-II e PGC-III), conforme detalhado na Tabela~\ref{tab:cronograma}.

\begin{table}[H]
\centering
\caption{Cronograma do Projeto}
\label{tab:cronograma}
\begin{tabular}{l|p{10cm}}
\hline
\textbf{Fase} & \textbf{Atividades Principais} \\ \hline
PGC-I (Fev–Abr/2025) & 
\begin{itemize}[leftmargin=*]
    \item Definição do tema e escopo da pesquisa
    \item Identificação de lacunas na literatura
    \item Execução das fases da RSL (busca, leitura e seleção dos artigos)
    \item Elaboração do relatório preliminar
\end{itemize} \\
PGC-II (Jun–Ago/2025) & 
\begin{itemize}[leftmargin=*]
    \item Leitura completa e análise dos artigos selecionados
    \item Extração e categorização das recomendações identificadas
    \item Elaboração de tabela comparativa das recomendações
    \item Elaboração do relatório intermediário
\end{itemize} \\
PGC-III (Set–Dez/2025) & 
\begin{itemize}[leftmargin=*]
    \item Consolidação dos resultados e análise
    \item Elaboração do relatório final
    \item Defesa do trabalho
\end{itemize} \\ \hline
\end{tabular}
\end{table}
