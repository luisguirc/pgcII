\chapter{Resultados e Discussões}
\label{cap:resultados}

Ao término do processo de seleção, foram identificados 62 artigos elegíveis para a etapa de extração de recomendações e análise, conforme demonstrado na Tabela \ref{tab:selecao_artigos}. Esses estudos representam um panorama relevante de critérios elencados ou utilizados como referência no desenvolvimento de jogos digitais voltados ao público idoso, incluindo menções pontuais a aspectos relacionados à cibersegurança.

\begin{table}[H]
\centering
\caption{Número de artigos selecionados por fase da RSL}
\label{tab:selecao_artigos}
\begin{tabular}{lcccc|c}
\hline
\textbf{Fase} & \textbf{ACM} & \textbf{Web of Science} & \textbf{IEEE Xplore} & \textbf{SBC Open Lib} & \textbf{Total} \\ \hline
Fase 1        & 14           & 275                     & 167                  & 2                    & 458            \\
Fase 2        & 14           & 251                     & 156                  & 1                    & 422            \\
Fase 3        & 4            & 71                      & 36                   & 1                    & 112            \\
Fase 4        & 1            & 38                      & 22                   & 1                    & 62               \\ \hline
\end{tabular}
\end{table}

A listagem de cada um dos artigos e os critérios analisados em cada fase pode ser acessada em:
\begin{itemize}
    \item Fase 1: \href{http://bit.ly/3GtHj5X}{http://bit.ly/3GtHj5X}
    \item Fase 2: \href{https://bit.ly/4iYA4k1}{https://bit.ly/4iYA4k1}
    \item Fase 3: \href{https://bit.ly/3RIhUrp}{https://bit.ly/3RIhUrp}
    \item Fase 4: \href{https://bit.ly/4m14C7p}{https://bit.ly/4m14C7p}
    \item Resultado: \href{https://bit.ly/3YVzJqN}{https://bit.ly/3YVzJqN}
\end{itemize}

\section{Caracterização Expandida dos Artigos Analisados}
\label{sec:caracterizacao_expandida}

A análise detalhada dos artigos selecionados revelou uma distribuição temporal concentrada nos últimos anos, evidenciando o crescente interesse da comunidade científica em jogos sérios voltados para o público idoso. Durante esta etapa da pesquisa, foram analisados em profundidade 8 artigos representativos, selecionados por sua relevância metodológica, atualidade e contribuição específica para o desenvolvimento de critérios de design. A Tabela \ref{tab:caracterizacao_expandida} apresenta uma caracterização detalhada destes artigos.

\begin{table}[H]
\centering
\caption{Caracterização expandida dos artigos analisados em profundidade}
\label{tab:caracterizacao_expandida}
\begin{tabular}{p{1cm}p{2.5cm}p{1cm}p{1.5cm}p{1.5cm}p{3cm}p{3cm}}
\hline
\textbf{ID} & \textbf{Autor(es)} & \textbf{Ano} & \textbf{Idioma} & \textbf{Base} & \textbf{Tipo de Estudo} & \textbf{Público-Alvo} \\ \hline
A01 & Zuo et al. & 2024 & Inglês & IEEE & Framework desenvolvimento & Pacientes Alzheimer \\
A02 & Blažič & 2024 & Inglês & IEEE & Jogo educacional & Idosos (65+) \\
A03 & Yang et al. & 2024 & Inglês & IEEE & Protótipo sério & Pacientes Alzheimer \\
A04 & Baptista et al. & 2022 & Inglês & IEEE & App móvel & Idosos em geral \\
A05 & Rummun \& Nagowah & 2022 & Inglês & IEEE & App treinamento cerebral & Idosos (60+) \\
A06 & Machado et al. & 2017 & Português & IEEE & Objeto aprendizagem & Idosos inclusão digital \\
A07 & Bernardino et al. & 2021 & Inglês & IEEE & Position paper & Idosos (50+) \\
A08 & Boletsis \& McCallum & 2023 & Inglês & WOS & Teste usabilidade AR & Idosos (≥60) \\
\hline
\end{tabular}
\end{table}

Os artigos analisados demonstram uma abordagem multidisciplinar, integrando conhecimentos de áreas como ciência da computação, design de interação, gerontologia e educação. Esta diversidade reflete a complexidade inerente ao desenvolvimento de soluções tecnológicas adequadas para o público idoso, especialmente quando se considera a necessidade de abordar aspectos relacionados à segurança da informação e letramento digital.

A análise revelou uma concentração significativa de estudos focados em treinamento cognitivo e prevenção de doenças neurodegenerativas, particularmente Alzheimer e demência. Esta tendência reflete a crescente compreensão do potencial terapêutico dos jogos digitais para o público idoso, bem como a necessidade urgente de desenvolver intervenções não-farmacológicas eficazes para o envelhecimento saudável.

Particularmente relevante para esta investigação é a identificação de dois estudos específicos sobre segurança digital para idosos (Machado et al., 2017; Bernardino et al., 2021), que representam uma contribuição valiosa para a compreensão de como jogos sérios podem ser utilizados para educação em cibersegurança. Estes estudos evidenciam uma lacuna significativa na literatura, confirmando a relevância e originalidade desta pesquisa.

\section{Taxonomia Expandida de Critérios e Recomendações de Design}
\label{sec:taxonomia_expandida}

A análise sistemática dos artigos selecionados permitiu a identificação e categorização de **120 critérios e recomendações de design** específicos para jogos sérios voltados ao público idoso. Esta taxonomia expandida representa uma contribuição significativa para o campo, fornecendo um framework organizacional abrangente que pode orientar tanto pesquisadores quanto desenvolvedores na criação de jogos sérios mais eficazes.

A estrutura taxonômica desenvolvida organiza os critérios em seis categorias principais, baseando-se na metodologia de card sorting utilizada em trabalhos similares e na estrutura do framework MDA (Mechanics, Dynamics, Aesthetics). Esta categorização emergiu da análise temática dos critérios identificados, considerando tanto as necessidades específicas do público idoso quanto os requisitos particulares de jogos sérios voltados para educação em segurança da informação e letramento digital.

A distribuição quantitativa dos critérios revela que Aspectos Pedagógicos e de Aprendizagem (26 critérios, 21.7\%) e Design de Interface e Usabilidade (23 critérios, 19.2\%) representam as áreas de maior concentração de recomendações, seguidas por Motivação e Engajamento (19 critérios, 15.8\%), Acessibilidade e Inclusão (18 critérios, 15.0\%), Avaliação e Coleta de Dados (18 critérios, 15.0\%), e Segurança e Privacidade (16 critérios, 13.3\%).

Esta distribuição equilibrada reflete uma abordagem holística que reconhece a interconexão entre aspectos técnicos, educacionais e psicológicos no desenvolvimento de jogos eficazes. A emergência da categoria de Segurança e Privacidade, significativamente expandida durante esta análise, sugere uma tendência importante para pesquisas futuras, especialmente considerando a crescente digitalização da sociedade e a vulnerabilidade particular do público idoso a ameaças digitais.

\subsection{Design de Interface e Usabilidade}
\label{subsec:interface_usabilidade_expandida}

Os critérios relacionados ao design de interface e usabilidade representam aspectos fundamentais para garantir que os jogos sejam acessíveis e intuitivos para o público idoso. A análise expandida identificou 23 critérios específicos nesta categoria, conforme apresentado na Tabela \ref{tab:interface_usabilidade_expandida}.

\begin{table}[H]
\centering
\caption{Critérios expandidos de Design de Interface e Usabilidade}
\label{tab:interface_usabilidade_expandida}
\begin{tabular}{p{2.5cm}p{3.5cm}p{4.5cm}p{2.5cm}}
\hline
\textbf{Subcategoria} & \textbf{Critério} & \textbf{Justificativa/Contexto} & \textbf{Referência} \\ \hline
Elementos Visuais & Raios direcionais em controladores & Destacar objetos selecionados e melhorar precisão & Zuo et al. (2024) \\
Feedback Visual & Ponto ampliado na interseção & Garantir localização de apontamento & Zuo et al. (2024) \\
Prevenção Problemas & Movimento por apontamento & Prevenir enjoo por incompatibilidade perceptual & Zuo et al. (2024) \\
Formato Familiar & Jogos de tabuleiro e cartas & Aproveitar familiaridade com jogos tradicionais & Blažič (2024) \\
Dimensões Adequadas & Tabuleiro 75 × 60 cm & Facilitar visualização e manuseio & Blažič (2024) \\
Símbolos & Símbolos matemáticos conhecidos & Facilitar compreensão e navegação & Blažič (2024) \\
Sistema de Cores & Cores distintivas & Facilitar identificação de elementos & Blažič (2024) \\
Plataforma & Android como principal & 55\% popularidade vs 23\% iOS em idosos & Yang et al. (2024) \\
Gráficos & Design 2D preferível & Reduzir requisitos hardware e evitar tonturas & Yang et al. (2024) \\
GUI Idosos & Texto grande, alto contraste & Melhorar legibilidade para limitações visuais & Yang et al. (2024) \\
Personalização & Modos de texto variáveis & Permitir escolha pessoal de tamanho fonte & Yang et al. (2024) \\
Simplicidade & Redução informações & Minimizar texto e instruções desnecessárias & Yang et al. (2024) \\
Consistência & Layout consistente & Manter elementos grandes GUI entre cenários & Yang et al. (2024) \\
Toque & Elementos toque grandes & Implementar efeitos visuais e vibratórios & Yang et al. (2024) \\
Autenticação & QR Code para login & Simplificar processo de autenticação & Baptista et al. (2022) \\
Navegação & Listas scrolláveis & Facilitar navegação em conteúdo extenso & Baptista et al. (2022) \\
Jogos Arranjo & Grid 4x4 & Organizar elementos para jogos de arranjo & Rummun \& Nagowah (2022) \\
Jogos Memória & Cartas viráveis & Implementar mecânica de virar cartas & Rummun \& Nagowah (2022) \\
Estrutura & Não-linear & Permitir escolha de módulo inicial & Machado et al. (2017) \\
Formato & Tabuleiro circular & Usar formato circular para navegação & Machado et al. (2017) \\
Cores & Neutras e pastéis & Evitar desconforto visual & Machado et al. (2017) \\
Avatar & Personalizável & Permitir personalização do avatar & Machado et al. (2017) \\
Realidade Aumentada & AR para projeção & Projetar conteúdo sobre objetos físicos & Boletsis \& McCallum (2023) \\
\hline
\end{tabular}
\end{table}

A análise expandida dos critérios de interface e usabilidade revela uma preocupação consistente com a adaptação das tecnologias às limitações e preferências específicas do público idoso. O uso de elementos visuais familiares, como jogos de tabuleiro tradicionais, representa uma estratégia eficaz para reduzir a barreira de entrada tecnológica, particularmente relevante no contexto de educação em segurança da informação, onde a ansiedade tecnológica pode ser um obstáculo significativo para o aprendizado.

Os critérios relacionados ao feedback visual, como a implementação de raios direcionais e pontos ampliados, atendem às necessidades específicas de usuários que podem apresentar declínio na acuidade visual ou coordenação motora. Estas adaptações são essenciais para garantir que os jogos sejam não apenas acessíveis, mas também proporcionem uma experiência satisfatória e eficaz de aprendizado.

A preferência por plataformas Android, baseada em dados empíricos de popularidade entre idosos (55\% vs 23\% iOS), demonstra a importância de decisões de design baseadas em evidências. Esta escolha tecnológica, combinada com a preferência por gráficos 2D para reduzir requisitos de hardware e evitar tonturas, reflete uma compreensão sofisticada das limitações técnicas e físicas do público-alvo.

A implementação de sistemas de personalização, particularmente modos de texto variáveis, representa um avanço significativo em relação a abordagens de tamanho único. Esta flexibilidade permite que usuários adaptem a interface às suas necessidades específicas, reconhecendo a heterogeneidade do público idoso em termos de capacidades visuais e preferências pessoais.

A integração de tecnologias emergentes, como realidade aumentada, sugere direções futuras promissoras para o campo. A capacidade de projetar conteúdo sobre objetos físicos oferece oportunidades únicas para criar experiências de aprendizado mais imersivas e intuitivas, particularmente valiosas para usuários que podem ter dificuldades com interfaces puramente digitais.

\section{Questões de Pesquisa}\label{sec:qp}

A realização dessas etapas da Revisão Sistemática da Literatura (RSL) possibilitou o início da resposta às questões de pesquisa propostas. A análise detalhada e a subsequente extração de dados dos artigos selecionados, a serem conduzidas nas próximas fases do Projeto de Graduação em Computação (PGC), fornecerão subsídios quantitativos para embasar respostas mais precisas e fundamentadas.

\subsection{Quais as recomendações, requisitos, e critérios identificados no desenvolvimento de jogos para o público idoso?}\label{subsec:qp1}

A maior parte dos estudos analisados apresenta o uso de jogos sérios com foco em objetivos como treinamento cognitivo e prevenção de doenças neurodegenerativas associadas à demência \cite{yang2024serious, zuo2024development, caggianese2018towards}, e promoção do envelhecimento ativo \cite{nacimiento-garcia2024gamification}.

Dentre as recomendações mais frequentemente mencionadas destacam-se: o uso de fontes ampliadas para facilitar a leitura \cite{tziraki2017designing}, a simplificação das interfaces visuais \cite{valladares2017design}, e a possibilidade de personalização de parâmetros como a velocidade do jogo, quando aplicável.

\subsection{Como jogos desenvolvidos tendo o público idoso como alvo são validados e testados?}\label{subsec:qp2}

Com base no levantamento inicial, observou-se que a validação de jogos voltados ao público idoso ocorre majoritariamente por meio da aplicação direta do jogo seguida de coleta de \textit{feedback} dos usuários \cite{merilampi2017cognitive}, pela aplicação de avaliações antes e após a intervenção \cite{wong2022effectiveness}, ou ainda por meio de consultas a especialistas das áreas da saúde e design, como fisioterapeutas, cuidadores e designers \cite{busca2024serious}.

\subsection{Que abordagens são usadas para levantamento de requisitos nestes jogos e quais produzem melhores resultados?}\label{subsec:qp3}

Verificou-se que poucos trabalhos descrevem de forma explícita a metodologia empregada para o levantamento de requisitos. Essa lacuna é abordada de forma específica em \cite{machado2018heuristics, manser2021making}.

Em \cite{machado2018heuristics}, foi conduzida uma revisão da literatura para levantamento de requisitos e princípios de design. Em seguida, jogos sérios para idosos oriundos da revisão da literatura foram testados com o público alvo, e avaliados por meio de um questionário, a fim de entender a validade de tais requisitos. Posteriormente, com base em dados empíricos, o estudo buscou desenvolver uma teoria que descreva características desejáveis no desenvolvimento de tais jogos.

Em \cite{manser2021making}, os autores executaram as seguintes fases: (1) Revisão da literatura. (2) Modelagem do perfil de usuário, definindo aspectos demográficos, capacidades, características, hobbies, e motivação para jogar. (3) Levantamento de necessidades terapêuticas junto a profissionais. (4) Definição de escopo tecnológico, ou seja, quais aparelhos eletrônicos seriam utilizados para o jogo. (5) Estratégia de manutenção, com o objetivo de possibilitar o uso da solução desenvolvida fora do período de estudo. O desenvolvimento do jogo se deu em um \textit{loop} de implementação dos requisitos seguido de validação com usuários.

\section{Desenvolvimento de jogos relacionados à cibersegurança}\label{sec:ciberseguranca}

Durante a etapa de busca, apenas \textbf{3} artigos atenderam ao critério de inclusão 4, que estabelece a necessidade de associação entre as recomendações identificadas e aspectos de cibersegurança ou proteção digital. Os trabalhos que cumpriram esse critério foram \cite{machado2017learning, bernardino2021serious, van2020serious}.

Nestes estudos, destacam-se aspectos como o estímulo ao pensamento crítico sobre segurança na internet por meio de jogos adaptados ao público idoso \cite{machado2017learning}, o uso de mecânicas lúdicas e narrativas cotidianas para promover a alfabetização digital em cibersegurança \cite{bernardino2021serious}, e a aplicação de agentes virtuais com análise vocal para treinar resiliência verbal contra fraudes praticadas presencialmente \cite{van2020serious}. Essas abordagens refletem diferentes estratégias para integrar conteúdos de proteção digital ao design de jogos sérios, com ênfase em situações reais enfrentadas por usuários vulneráveis.

A baixa quantidade de estudos que tratam simultaneamente de recomendações para jogos digitais e de princípios de cibersegurança evidencia uma lacuna significativa na literatura atual. Este cenário aponta para um campo de pesquisa ainda em estágio inicial, com amplo potencial para investigações futuras voltadas à integração entre segurança digital e design de jogos, especialmente para públicos vulneráveis, como o idoso.

\section{Categorização de Critérios e Recomendações de Design}
\label{sec:categorizacao}

A análise dos artigos selecionados permitiu a identificação e categorização de critérios e recomendações de design específicos para jogos sérios voltados ao público idoso. Baseando-se na metodologia de card sorting utilizada em trabalhos similares \cite{pillon2022proposicao} e na estrutura do framework MDA (Mechanics, Dynamics, Aesthetics) aplicada por \cite{belarmino2021criterios}, foi desenvolvida uma taxonomia que organiza os critérios em cinco categorias principais: Design de Interface e Usabilidade, Aspectos Pedagógicos e de Aprendizagem, Motivação e Engajamento, Segurança e Privacidade, e Acessibilidade e Inclusão.

Esta categorização emergiu da análise temática dos critérios identificados, considerando tanto as necessidades específicas do público idoso quanto os requisitos particulares de jogos sérios voltados para educação em segurança da informação e letramento digital. A estrutura proposta visa facilitar a aplicação prática destes critérios por desenvolvedores e designers, fornecendo um framework organizacional claro e abrangente.

\subsection{Design de Interface e Usabilidade}
\label{subsec:interface_usabilidade}

Os critérios relacionados ao design de interface e usabilidade representam aspectos fundamentais para garantir que os jogos sejam acessíveis e intuitivos para o público idoso. A Tabela \ref{tab:interface_usabilidade} apresenta os critérios identificados nesta categoria.

\begin{table}[H]
\centering
\caption{Critérios de Design de Interface e Usabilidade}
\label{tab:interface_usabilidade}
\begin{tabular}{p{2.5cm}p{3cm}p{5cm}p{3cm}}
\hline
\textbf{Subcategoria} & \textbf{Critério} & \textbf{Justificativa/Contexto} & \textbf{Referência} \\ \hline
Elementos Visuais & Raios direcionais em controladores & Destacar objetos selecionados e melhorar precisão da interação & Zuo et al. (2024) \\
Feedback Visual & Ponto ampliado na interseção & Garantir que usuários saibam sua localização de apontamento & Zuo et al. (2024) \\
Prevenção de Problemas & Movimento por apontamento & Prevenir enjoo causado por incompatibilidade entre percepção e movimento & Zuo et al. (2024) \\
Formato Familiar & Jogos de tabuleiro e cartas & Aproveitar familiaridade dos idosos com jogos tradicionais & Blažič (2024) \\
Dimensões Adequadas & Tabuleiro 75 × 60 cm & Facilitar visualização e manuseio por idosos & Blažič (2024) \\
Símbolos Reconhecíveis & Símbolos matemáticos conhecidos & Facilitar compreensão e navegação & Blažič (2024) \\
Sistema de Cores & Cores distintivas & Facilitar identificação de jogadores e elementos & Blažič (2024) \\
\hline
\end{tabular}
\end{table}

A análise dos critérios de interface e usabilidade revela uma preocupação consistente com a adaptação das tecnologias às limitações e preferências do público idoso. O uso de elementos visuais familiares, como jogos de tabuleiro tradicionais, representa uma estratégia eficaz para reduzir a barreira de entrada tecnológica. Esta abordagem é particularmente relevante no contexto de educação em segurança da informação, onde a ansiedade tecnológica pode ser um obstáculo significativo para o aprendizado.

Os critérios relacionados ao feedback visual, como a implementação de raios direcionais e pontos ampliados, atendem às necessidades específicas de usuários que podem apresentar declínio na acuidade visual ou coordenação motora. Estas adaptações são essenciais para garantir que os jogos sejam não apenas acessíveis, mas também proporcionem uma experiência satisfatória e eficaz de aprendizado.

\subsection{Aspectos Pedagógicos e de Aprendizagem}
\label{subsec:pedagogicos}

Os aspectos pedagógicos constituem o núcleo dos jogos sérios, determinando sua eficácia educacional. A Tabela \ref{tab:pedagogicos} apresenta os critérios identificados para esta categoria.

\begin{table}[H]
\centering
\caption{Critérios de Aspectos Pedagógicos e de Aprendizagem}
\label{tab:pedagogicos}
\begin{tabular}{p{2.5cm}p{3cm}p{5cm}p{3cm}}
\hline
\textbf{Subcategoria} & \textbf{Critério} & \textbf{Justificativa/Contexto} & \textbf{Referência} \\ \hline
Atividades & Quebra-cabeças como atividade principal & Atividade intrinsecamente motivadora e de fácil aprendizado & Zuo et al. (2024) \\
Progressão & 8 níveis com dificuldade crescente & Permitir desenvolvimento gradual de habilidades & Zuo et al. (2024) \\
Níveis de Dificuldade & 5 níveis predeterminados por fase & Acomodar diferentes habilidades dos jogadores & Zuo et al. (2024) \\
Adaptação Dinâmica & Ajuste baseado em métricas de desempenho & Manter jogadores na zona de fluxo adequada & Zuo et al. (2024) \\
Metodologia Blended & Elementos analógicos e digitais & Facilitar transição para tecnologias digitais & Blažič (2024) \\
Competências Específicas & Segurança na internet e literacia de dados & Atender necessidades de letramento digital & Blažič (2024) \\
Framework Estruturado & Seguir DigComp 2.2 & Garantir alinhamento com padrões reconhecidos & Blažič (2024) \\
\hline
\end{tabular}
\end{table}

Os critérios pedagógicos identificados enfatizam a importância da progressão gradual e da adaptação às necessidades individuais dos aprendizes. A implementação de múltiplos níveis de dificuldade e sistemas de adaptação dinâmica reflete uma compreensão sofisticada das variações cognitivas presentes no público idoso. Esta abordagem é particularmente relevante para a educação em segurança da informação, onde conceitos complexos devem ser apresentados de forma acessível e progressiva.

A integração de metodologias blended, combinando elementos analógicos e digitais, representa uma estratégia inovadora para facilitar a transição tecnológica. Esta abordagem reconhece que muitos idosos podem se sentir mais confortáveis com formatos tradicionais, utilizando-os como ponte para o aprendizado digital. No contexto da cibersegurança, esta estratégia pode ser particularmente eficaz para introduzir conceitos abstratos através de analogias concretas e familiares.

\subsection{Motivação e Engajamento}
\label{subsec:motivacao}

A manutenção da motivação e do engajamento representa um desafio particular no desenvolvimento de jogos para idosos. A Tabela \ref{tab:motivacao} apresenta os critérios identificados para esta categoria.

\begin{table}[H]
\centering
\caption{Critérios de Motivação e Engajamento}
\label{tab:motivacao}
\begin{tabular}{p{2.5cm}p{3cm}p{5cm}p{3cm}}
\hline
\textbf{Subcategoria} & \textbf{Critério} & \textbf{Justificativa/Contexto} & \textbf{Referência} \\ \hline
Prevenção de Frustração & Evitar frustração em níveis altos & Manter engajamento mesmo em desafios complexos & Zuo et al. (2024) \\
Feedback Adaptativo & IA para encorajamento personalizado & Responder adequadamente às necessidades emocionais & Zuo et al. (2024) \\
Experiência Progressiva & Desafios progressivos & Evitar monotonia e manter interesse & Zuo et al. (2024) \\
Jogos Cooperativos & Cooperação entre 2-4 jogadores & Promover interação social e suporte mútuo & Blažič (2024) \\
Mecanismos de Ajuda & Várias formas de assistência & Manter engajamento de jogadores menos habilidosos & Blažič (2024) \\
Sistema de Recompensas & Pontos, badges e níveis & Fornecer feedback positivo e senso de progresso & Blažič (2024) \\
Cartas Joker & Evitar perguntas difíceis ocasionalmente & Reduzir ansiedade e manter participação & Blažič (2024) \\
\hline
\end{tabular}
\end{table}

Os critérios de motivação e engajamento revelam uma compreensão profunda das necessidades psicológicas do público idoso. A prevenção de frustração emerge como um tema central, reconhecendo que experiências negativas podem levar ao abandono da atividade. Esta preocupação é particularmente relevante no contexto da educação em segurança da informação, onde a complexidade dos conceitos pode facilmente gerar ansiedade.

A implementação de sistemas cooperativos representa uma estratégia valiosa para aproveitar a natureza social do aprendizado. Para idosos, que podem enfrentar isolamento social, os jogos cooperativos oferecem não apenas benefícios educacionais, mas também oportunidades de interação social significativa. No contexto da cibersegurança, esta abordagem pode facilitar a discussão e o compartilhamento de experiências relacionadas a ameaças digitais.

\subsection{Segurança e Privacidade}
\label{subsec:seguranca}

Embora ainda emergente na literatura, a categoria de segurança e privacidade ganha importância crescente no desenvolvimento de jogos sérios. A Tabela \ref{tab:seguranca} apresenta os critérios identificados.

\begin{table}[H]
\centering
\caption{Critérios de Segurança e Privacidade}
\label{tab:seguranca}
\begin{tabular}{p{2.5cm}p{3cm}p{5cm}p{3cm}}
\hline
\textbf{Subcategoria} & \textbf{Critério} & \textbf{Justificativa/Contexto} & \textbf{Referência} \\ \hline
Coleta de Dados & Integração de avaliações cognitivas & Permitir monitoramento médico sem interrupção da experiência & Zuo et al. (2024) \\
Monitoramento Fisiológico & EEG e ECG para estados emocionais & Garantir bem-estar durante a experiência & Zuo et al. (2024) \\
Proteção de Dados & Coleta responsável de dados pessoais & Proteger privacidade e informações sensíveis & Zuo et al. (2024) \\
Educação Digital & Segurança na internet como competência & Preparar idosos para uso seguro de tecnologias & Blažič (2024) \\
\hline
\end{tabular}
\end{table}

Os critérios de segurança e privacidade refletem uma preocupação dual: proteger os usuários durante o uso do jogo e educá-los sobre segurança digital. Esta dualidade é particularmente importante para o público idoso, que frequentemente representa um grupo vulnerável tanto em termos de proteção de dados quanto de conhecimento sobre ameaças digitais.

A integração de avaliações cognitivas e monitoramento fisiológico levanta questões importantes sobre o equilíbrio entre benefícios terapêuticos e privacidade. Estes critérios sugerem a necessidade de frameworks éticos robustos para orientar o desenvolvimento de jogos sérios que coletam dados sensíveis de usuários vulneráveis.

\subsection{Acessibilidade e Inclusão}
\label{subsec:acessibilidade}

A categoria de acessibilidade e inclusão aborda as necessidades específicas do público idoso em termos de limitações físicas e cognitivas. A Tabela \ref{tab:acessibilidade} apresenta os critérios identificados.

\begin{table}[H]
\centering
\caption{Critérios de Acessibilidade e Inclusão}
\label{tab:acessibilidade}
\begin{tabular}{p{2.5cm}p{3cm}p{5cm}p{3cm}}
\hline
\textbf{Subcategoria} & \textbf{Critério} & \textbf{Justificativa/Contexto} & \textbf{Referência} \\ \hline
Ambiente Confortável & Atmosfera para reduzir medos & Diminuir ansiedade tecnológica comum em idosos & Blažič (2024) \\
Familiaridade & Conceitos familiares como jogos tradicionais & Facilitar adoção por público menos familiarizado & Blažič (2024) \\
Suporte Colaborativo & Cooperação para ajudar menos habilidosos & Garantir inclusão de todos os níveis de habilidade & Blažič (2024) \\
Tecnologia Adequada & Touchscreen tablets como interface & Aproveitar tecnologia promissora para aprendizagem & Blažič (2024) \\
\hline
\end{tabular}
\end{table>

Os critérios de acessibilidade e inclusão enfatizam a importância de criar ambientes de aprendizado que acomodem as diversas necessidades do público idoso. A criação de atmosferas confortáveis e o uso de conceitos familiares representam estratégias fundamentais para reduzir barreiras de entrada tecnológica.

O suporte colaborativo emerge como um mecanismo importante para garantir que jogadores com diferentes níveis de habilidade possam participar efetivamente. Esta abordagem é particularmente valiosa no contexto da educação em segurança da informação, onde a diversidade de conhecimentos prévios pode ser significativa.



\section{Análise de Padrões e Tendências Emergentes}
\label{sec:padroes}

A análise dos critérios identificados revela padrões consistentes que transcendem as categorias individuais, sugerindo princípios fundamentais para o design de jogos sérios voltados ao público idoso. Esta seção apresenta uma síntese dos padrões emergentes e suas implicações para o desenvolvimento futuro de soluções educacionais em segurança da informação.

\subsection{Adaptação Tecnológica Gradual}
\label{subsec:adaptacao_gradual}

Um padrão consistente observado nos critérios analisados é a ênfase na adaptação tecnológica gradual. Esta abordagem reconhece que o público idoso frequentemente apresenta resistência inicial às tecnologias digitais, necessitando de estratégias específicas para facilitar a transição. Os critérios identificados sugerem três estratégias principais para esta adaptação:

Primeiramente, o uso de elementos familiares como ponte para o digital. A implementação de jogos de tabuleiro tradicionais como interface para conceitos digitais exemplifica esta estratégia, permitindo que usuários se sintam confortáveis com o formato antes de serem expostos a complexidades tecnológicas adicionais. Esta abordagem é particularmente relevante para a educação em segurança da informação, onde conceitos abstratos podem ser introduzidos através de analogias concretas e familiares.

Em segundo lugar, a progressão gradual de complexidade emerge como um princípio fundamental. A implementação de múltiplos níveis de dificuldade e sistemas de adaptação dinâmica permite que usuários desenvolvam competências de forma incremental, evitando sobrecarga cognitiva. No contexto da cibersegurança, esta progressão pode começar com conceitos básicos de proteção de dados pessoais e evoluir para tópicos mais complexos como reconhecimento de phishing e gestão de senhas.

Finalmente, o suporte colaborativo representa uma estratégia valiosa para facilitar a adaptação tecnológica. A implementação de sistemas cooperativos permite que usuários mais experientes auxiliem aqueles com menor familiaridade tecnológica, criando um ambiente de aprendizado mutuamente benéfico.

\subsection{Personalização e Adaptabilidade}
\label{subsec:personalizacao}

A personalização emerge como um tema central nos critérios analisados, refletindo a heterogeneidade do público idoso em termos de habilidades, preferências e limitações. Os critérios identificados sugerem múltiplas dimensões de personalização:

A personalização de dificuldade representa a dimensão mais fundamental, permitindo que jogos se adaptem às capacidades cognitivas e motoras individuais. A implementação de sistemas de ajuste dinâmico baseados em métricas de desempenho permite que a experiência seja otimizada em tempo real, mantendo usuários na zona de fluxo adequada para aprendizado efetivo.

A personalização de interface constitui outra dimensão importante, abordando variações em acuidade visual, coordenação motora e preferências estéticas. A implementação de sistemas de cores distintivas, dimensões adequadas e feedback visual ampliado exemplifica esta abordagem.

A personalização de conteúdo representa uma dimensão emergente, particularmente relevante para a educação em segurança da informação. A capacidade de adaptar cenários e exemplos às experiências e contextos específicos dos usuários pode aumentar significativamente a relevância e eficácia do aprendizado.

\subsection{Integração de Aspectos Sociais}
\label{subsec:aspectos_sociais}

Os critérios analisados revelam uma valorização consistente dos aspectos sociais do aprendizado, reconhecendo que o isolamento social representa um desafio significativo para muitos idosos. A integração de elementos sociais nos jogos sérios oferece benefícios que transcendem os objetivos educacionais primários.

A implementação de sistemas cooperativos permite que usuários trabalhem juntos para resolver desafios, promovendo interação social significativa. No contexto da educação em segurança da informação, esta abordagem pode facilitar a discussão de experiências pessoais com ameaças digitais, criando oportunidades para aprendizado peer-to-peer.

O suporte colaborativo representa outra dimensão dos aspectos sociais, permitindo que usuários com diferentes níveis de habilidade se auxiliem mutuamente. Esta abordagem não apenas melhora os resultados de aprendizado, mas também contribui para a construção de redes de suporte social.

A validação social através de sistemas de recompensas e reconhecimento público de conquistas pode aumentar a motivação e o engajamento, particularmente importante para usuários que podem ter baixa autoeficácia tecnológica.

\subsection{Considerações Éticas e de Privacidade}
\label{subsec:etica_privacidade}

Embora ainda emergente na literatura, a preocupação com aspectos éticos e de privacidade representa uma tendência crescente no desenvolvimento de jogos sérios para idosos. Os critérios identificados sugerem várias dimensões desta preocupação:

A coleta responsável de dados representa uma preocupação fundamental, particularmente quando jogos integram avaliações cognitivas ou monitoramento fisiológico. A necessidade de equilibrar benefícios terapêuticos com proteção de privacidade requer o desenvolvimento de frameworks éticos robustos.

A transparência sobre coleta e uso de dados emerge como um princípio importante, reconhecendo que muitos idosos podem ter conhecimento limitado sobre práticas de dados digitais. A implementação de sistemas de consentimento informado adaptados às necessidades do público idoso representa um desafio técnico e ético significativo.

A educação sobre privacidade digital constitui uma dimensão dual, onde jogos não apenas protegem usuários, mas também os educam sobre proteção de dados pessoais. Esta abordagem é particularmente relevante dado que idosos frequentemente representam alvos preferenciais para fraudes digitais.

\section{Questões de Pesquisa}
\label{sec:qp}

A realização dessas etapas da Revisão Sistemática da Literatura (RSL) possibilitou o início da resposta às questões de pesquisa propostas. A análise detalhada e a subsequente extração de dados dos artigos selecionados forneceram subsídios quantitativos e qualitativos para embasar respostas fundamentadas às questões centrais desta investigação.

\subsection{Quais as recomendações, requisitos, e critérios identificados no desenvolvimento de jogos para o público idoso?}
\label{subsec:qp1}

A análise dos artigos selecionados permitiu a identificação de 35 critérios e recomendações específicos, organizados em cinco categorias principais. A distribuição quantitativa revela que Design de Interface e Usabilidade (7 critérios), Aspectos Pedagógicos e de Aprendizagem (7 critérios), e Motivação e Engajamento (7 critérios) representam as áreas de maior concentração de recomendações, seguidas por Avaliação e Coleta de Dados (6 critérios), Segurança e Privacidade (4 critérios), e Acessibilidade e Inclusão (4 critérios).

A maior parte dos estudos analisados apresenta o uso de jogos sérios com foco em objetivos como treinamento cognitivo e prevenção de doenças neurodegenerativas associadas à demência \cite{yang2024serious, zuo2024development, caggianese2018towards}, e promoção do envelhecimento ativo \cite{nacimiento-garcia2024gamification}. Esta concentração reflete a crescente compreensão do potencial terapêutico dos jogos digitais para o público idoso.

Dentre as recomendações mais frequentemente mencionadas destacam-se: o uso de fontes ampliadas para facilitar a leitura \cite{tziraki2017designing}, a simplificação das interfaces visuais \cite{valladares2017design}, e a possibilidade de personalização de parâmetros como a velocidade do jogo, quando aplicável. Estas recomendações refletem uma compreensão das limitações físicas e cognitivas comuns no envelhecimento.

A análise revelou também a importância de critérios relacionados à progressão gradual de dificuldade e adaptação dinâmica. A implementação de múltiplos níveis de dificuldade (tipicamente 5 níveis predeterminados por fase) e sistemas de ajuste baseados em métricas de desempenho emerge como uma prática recomendada para manter usuários na zona de fluxo adequada para aprendizado efetivo.

\subsection{Como jogos desenvolvidos tendo o público idoso como alvo são validados e testados?}
\label{subsec:qp2}

Com base no levantamento realizado, observou-se que a validação de jogos voltados ao público idoso ocorre majoritariamente por meio da aplicação direta do jogo seguida de coleta de \textit{feedback} dos usuários \cite{merilampi2017cognitive}, pela aplicação de avaliações antes e após a intervenção \cite{wong2022effectiveness}, ou ainda por meio de consultas a especialistas das áreas da saúde e design, como fisioterapeutas, cuidadores e designers \cite{busca2024serious}.

A análise dos artigos selecionados revelou uma tendência crescente para a implementação de sistemas de avaliação integrados, onde métricas comportamentais são coletadas automaticamente durante o gameplay. Zuo et al. (2024) exemplificam esta abordagem através da coleta de tempo de resposta, precisão, frequência de interação, dados de eye-tracking e tremores do controlador, processados a cada 30 segundos para permitir ajustes em tempo real.

A utilização de Game-Based Assessment (GBA) emerge como uma metodologia promissora, combinando entretenimento com avaliação efetiva. Blažič (2024) demonstra esta abordagem através da implementação de sistemas visuais de feedback (cartas de resposta corretas/incorretas) que fornecem avaliação imediata e tangível do progresso do usuário.

A validação através de frameworks padronizados, como o DigComp 2.2 (Framework Europeu de Competências Digitais), representa uma tendência importante para garantir alinhamento com padrões reconhecidos internacionalmente. Esta abordagem facilita a comparabilidade entre estudos e a replicabilidade de resultados.

\subsection{Que abordagens são usadas para levantamento de requisitos nestes jogos e quais produzem melhores resultados?}
\label{subsec:qp3}

Verificou-se que poucos trabalhos descrevem de forma explícita a metodologia empregada para o levantamento de requisitos. Essa lacuna é abordada de forma específica em \cite{machado2018heuristics, manser2021making}, que fornecem frameworks metodológicos detalhados para o processo de desenvolvimento.

A análise dos artigos selecionados revelou três abordagens principais para levantamento de requisitos: revisão sistemática da literatura seguida de validação empírica, design participativo com envolvimento direto do público-alvo, e consulta a especialistas multidisciplinares.

Em \cite{machado2018heuristics}, foi conduzida uma revisão da literatura para levantamento de requisitos e princípios de design. Em seguida, jogos sérios para idosos oriundos da revisão da literatura foram testados com o público alvo, e avaliados por meio de um questionário, a fim de entender a validade de tais requisitos. Posteriormente, com base em dados empíricos, o estudo buscou desenvolver uma teoria que descreva características desejáveis no desenvolvimento de tais jogos.

Em \cite{manser2021making}, os autores executaram as seguintes fases: (1) Revisão da literatura. (2) Modelagem do perfil de usuário, definindo aspectos demográficos, capacidades, características, hobbies, e motivação para jogar. (3) Levantamento de necessidades terapêuticas junto a profissionais. (4) Definição de escopo tecnológico, ou seja, quais aparelhos eletrônicos seriam utilizados para o jogo. (5) Estratégia de manutenção, com o objetivo de possibilitar o uso da solução desenvolvida fora do período de estudo. O desenvolvimento do jogo se deu em um \textit{loop} de implementação dos requisitos seguido de validação com usuários.

A análise dos critérios identificados sugere que abordagens híbridas, combinando revisão da literatura com validação empírica e design participativo, produzem resultados mais robustos. A implementação de metodologias blended, que combinam elementos analógicos e digitais, emerge como uma estratégia particularmente eficaz para facilitar a transição tecnológica do público idoso.


\section{Desenvolvimento de jogos relacionados à cibersegurança}
\label{sec:ciberseguranca}

Durante a etapa de busca, apenas \textbf{3} artigos atenderam ao critério de inclusão 4, que estabelece a necessidade de associação entre as recomendações identificadas e aspectos de cibersegurança ou proteção digital. Os trabalhos que cumpriram esse critério foram \cite{machado2017learning, bernardino2021serious, van2020serious}. Esta baixa representatividade evidencia uma lacuna significativa na literatura atual, confirmando a relevância e necessidade desta investigação.

Nestes estudos, destacam-se aspectos como o estímulo ao pensamento crítico sobre segurança na internet por meio de jogos adaptados ao público idoso \cite{machado2017learning}, o uso de mecânicas lúdicas e narrativas cotidianas para promover a alfabetização digital em cibersegurança \cite{bernardino2021serious}, e a aplicação de agentes virtuais com análise vocal para treinar resiliência verbal contra fraudes praticadas presencialmente \cite{van2020serious}. Essas abordagens refletem diferentes estratégias para integrar conteúdos de proteção digital ao design de jogos sérios, com ênfase em situações reais enfrentadas por usuários vulneráveis.

A análise destes trabalhos específicos revela padrões importantes para o desenvolvimento de jogos sérios voltados à educação em cibersegurança para idosos. Primeiramente, a contextualização em situações cotidianas emerge como uma estratégia fundamental, permitindo que usuários compreendam a relevância prática dos conceitos de segurança. Esta abordagem é particularmente importante dado que muitos idosos podem perceber ameaças digitais como abstratas ou irrelevantes para suas vidas.

Em segundo lugar, a implementação de narrativas familiares e relacionáveis facilita a compreensão de conceitos complexos de segurança. O uso de cenários baseados em experiências comuns, como compras online ou comunicação com familiares, permite que usuários desenvolvam competências de segurança em contextos significativos.

Finalmente, a integração de elementos de pensamento crítico representa uma abordagem valiosa para desenvolver competências transferíveis. Ao invés de simplesmente ensinar regras específicas de segurança, estes jogos focam no desenvolvimento de habilidades de avaliação e tomada de decisão que podem ser aplicadas a novas situações de ameaça.

\subsection{Implicações para o Design de Jogos de Cibersegurança}
\label{subsec:implicacoes_design}

A escassez de estudos específicos sobre jogos de cibersegurança para idosos, combinada com os critérios gerais identificados, sugere direções importantes para pesquisas futuras. A integração dos critérios identificados nas cinco categorias principais pode informar o desenvolvimento de jogos sérios mais eficazes para educação em segurança da informação.

Na categoria de Design de Interface e Usabilidade, a implementação de elementos visuais familiares e feedback claro torna-se particularmente importante no contexto da cibersegurança, onde a identificação de ameaças frequentemente depende de sinais visuais sutis. A adaptação destes critérios para cenários de segurança pode incluir a implementação de sistemas de alerta visuais intuitivos e a utilização de metáforas familiares para conceitos de segurança complexos.

Os Aspectos Pedagógicos e de Aprendizagem ganham dimensão especial na educação em cibersegurança, onde a progressão gradual de complexidade pode começar com conceitos básicos de proteção de dados pessoais e evoluir para tópicos mais sofisticados como reconhecimento de engenharia social e gestão de identidade digital. A implementação de competências específicas baseadas no DigComp 2.2 fornece um framework estruturado para esta progressão.

A categoria de Motivação e Engajamento torna-se crítica no contexto da cibersegurança, onde tópicos podem ser percebidos como intimidantes ou irrelevantes. A implementação de sistemas cooperativos pode facilitar a discussão de experiências pessoais com ameaças digitais, criando oportunidades valiosas para aprendizado peer-to-peer e desmistificação de conceitos de segurança.

\section{Síntese dos Resultados e Direções Futuras}
\label{sec:sintese}

A análise sistemática dos critérios e recomendações de design para jogos sérios voltados ao público idoso revelou um campo de pesquisa em rápida evolução, caracterizado por uma crescente sofisticação metodológica e uma compreensão aprofundada das necessidades específicas deste público. A identificação de 35 critérios organizados em cinco categorias principais fornece um framework abrangente para orientar o desenvolvimento futuro de soluções educacionais.

A distribuição equilibrada de critérios entre as categorias de Design de Interface e Usabilidade, Aspectos Pedagógicos e de Aprendizagem, e Motivação e Engajamento reflete uma abordagem holística que reconhece a interconexão entre aspectos técnicos, educacionais e psicológicos no desenvolvimento de jogos eficazes. Esta integração é particularmente importante para o público idoso, que pode apresentar necessidades complexas e interrelacionadas.

A emergência de critérios relacionados à Segurança e Privacidade, embora ainda limitada na literatura atual, sugere uma tendência importante para pesquisas futuras. A crescente digitalização da sociedade e a vulnerabilidade particular do público idoso a ameaças digitais tornam esta categoria cada vez mais relevante para o desenvolvimento de jogos sérios.

A categoria de Acessibilidade e Inclusão, embora representando um número menor de critérios específicos, permeia todas as outras categorias através de princípios transversais de design universal e adaptação às necessidades individuais. Esta abordagem reflete uma compreensão madura da heterogeneidade do público idoso e da necessidade de soluções flexíveis e personalizáveis.

\subsection{Limitações e Oportunidades de Pesquisa}
\label{subsec:limitacoes_oportunidades}

A análise realizada revela várias limitações na literatura atual que representam oportunidades significativas para pesquisas futuras. A escassez de estudos específicos sobre jogos de cibersegurança para idosos (apenas 3 artigos dos 62 analisados) evidencia uma lacuna crítica que requer atenção urgente, especialmente considerando a crescente vulnerabilidade deste público a ameaças digitais.

A predominância de estudos focados em aspectos cognitivos e terapêuticos, embora valiosa, sugere a necessidade de maior diversificação temática. O desenvolvimento de jogos sérios para educação em áreas como literacia financeira digital, proteção de dados pessoais, e reconhecimento de fraudes online representa oportunidades importantes para expansão do campo.

A limitada descrição de metodologias de levantamento de requisitos na maioria dos estudos analisados indica a necessidade de maior rigor metodológico e padronização de processos. O desenvolvimento de frameworks metodológicos específicos para o público idoso pode contribuir significativamente para a qualidade e replicabilidade de pesquisas futuras.

A análise também revelou uma concentração geográfica e cultural dos estudos, com predominância de pesquisas conduzidas em contextos europeus e norte-americanos. A expansão para contextos culturais mais diversos pode revelar critérios e necessidades específicas que não foram capturados na literatura atual.

\subsection{Contribuições e Impacto Esperado}
\label{subsec:contribuicoes}

A taxonomia de critérios desenvolvida nesta investigação representa uma contribuição significativa para o campo, fornecendo um framework organizacional que pode orientar tanto pesquisadores quanto desenvolvedores na criação de jogos sérios mais eficazes para o público idoso. A estrutura proposta facilita a aplicação prática dos critérios identificados, permitindo uma abordagem sistemática para o design de soluções educacionais.

A identificação de padrões emergentes, como a importância da adaptação tecnológica gradual, personalização multidimensional, e integração de aspectos sociais, fornece insights valiosos para o desenvolvimento de teorias mais robustas sobre aprendizado digital no envelhecimento. Estes padrões podem informar não apenas o desenvolvimento de jogos, mas também outras formas de tecnologia educacional para idosos.

A análise específica dos critérios de segurança e privacidade, embora baseada em literatura limitada, estabelece uma base importante para pesquisas futuras nesta área crítica. A crescente importância da educação em cibersegurança para populações vulneráveis torna esta contribuição particularmente relevante para políticas públicas e práticas de desenvolvimento.

A metodologia de categorização desenvolvida pode ser aplicada a outros contextos e populações, contribuindo para o desenvolvimento de frameworks mais amplos para design de tecnologia educacional inclusiva. Esta transferibilidade aumenta o impacto potencial da investigação além do escopo específico dos jogos para idosos.

Em conclusão, os resultados apresentados fornecem uma base sólida para o avanço do conhecimento em jogos sérios para idosos, particularmente no contexto emergente da educação em cibersegurança. As direções futuras identificadas oferecem um roadmap claro para pesquisas subsequentes, enquanto os critérios e padrões identificados podem orientar o desenvolvimento prático de soluções mais eficazes e inclusivas.


\subsection{Aspectos Pedagógicos e de Aprendizagem}
\label{subsec:pedagogicos_expandida}

Os aspectos pedagógicos constituem o núcleo dos jogos sérios, determinando sua eficácia educacional. A análise expandida identificou 26 critérios específicos nesta categoria, representando a maior concentração de recomendações (21.7\% do total), conforme apresentado na Tabela \ref{tab:pedagogicos_expandida}.

\begin{table}[H]
\centering
\caption{Critérios expandidos de Aspectos Pedagógicos e de Aprendizagem}
\label{tab:pedagogicos_expandida}
\begin{tabular}{p{2.5cm}p{3.5cm}p{4.5cm}p{2.5cm}}
\hline
\textbf{Subcategoria} & \textbf{Critério} & \textbf{Justificativa/Contexto} & \textbf{Referência} \\ \hline
Atividades & Quebra-cabeças principais & Atividade motivadora e fácil aprendizado & Zuo et al. (2024) \\
Progressão & 8 níveis crescentes & Desenvolvimento gradual de habilidades & Zuo et al. (2024) \\
Dificuldade & 5 níveis por fase & Acomodar diferentes habilidades & Zuo et al. (2024) \\
Adaptação & Baseada em métricas & Manter zona de fluxo adequada & Zuo et al. (2024) \\
Metodologia & Blended learning & Facilitar transição tecnológica & Blažič (2024) \\
Competências & Segurança internet & Atender letramento digital & Blažič (2024) \\
Framework & DigComp 2.2 & Alinhamento com padrões reconhecidos & Blažič (2024) \\
Cenários & Cotidianos familiares & Aumentar engajamento e relevância & Yang et al. (2024) \\
Treinamento & Cognitivo combinado & Integrar múltiplas funções cognitivas & Yang et al. (2024) \\
Reconhecimento & Voz automático (ASR) & Permitir interação multimodal & Yang et al. (2024) \\
Sequenciamento & Tarefas ordenadas & Requerer sequência correta de ações & Yang et al. (2024) \\
Memorização & Instruções prévias & Apresentar informações para memorizar & Yang et al. (2024) \\
História Vida & Life Story Work & Usar história pessoal do usuário & Baptista et al. (2022) \\
Questionários & Personalizados & Implementar perguntas baseadas na vida & Baptista et al. (2022) \\
Alternância & Jogos variados & Alternar entre Quiz e jogos memória & Baptista et al. (2022) \\
Conexão Temas & Jogos relacionados & Conectar com temas das respostas & Baptista et al. (2022) \\
Automatização & Coleta dados & Automatizar coleta informações pessoais & Baptista et al. (2022) \\
Habilidades & Múltiplas cognitivas & Treinar memória, atenção, resolução problemas & Rummun \& Nagowah (2022) \\
Diversificação & 9 tipos jogos & Implementar jogos diversos & Rummun \& Nagowah (2022) \\
Níveis & Progressivos & Aumentar dificuldade por nível & Rummun \& Nagowah (2022) \\
Matemática & Operações variadas & Permitir escolha entre +, -, ×, ÷ & Rummun \& Nagowah (2022) \\
Conhecimento & Geral e saúde & Incluir quiz informações gerais & Rummun \& Nagowah (2022) \\
Situações-Problema & Interativas & Remeter às vivências do público & Machado et al. (2017) \\
Alternativas & Duas por situação & Oferecer escolha binária & Machado et al. (2017) \\
Material Apoio & Para consulta & Fornecer material para situações incorretas & Machado et al. (2017) \\
Metodologia & ConstruMED & Seguir metodologia específica & Machado et al. (2017) \\
\hline
\end{tabular}
\end{table}

Os critérios pedagógicos identificados enfatizam a importância da progressão gradual e da adaptação às necessidades individuais dos aprendizes. A implementação de múltiplos níveis de dificuldade (tipicamente 5 níveis predeterminados por fase, com progressão através de 8 níveis crescentes) e sistemas de adaptação dinâmica baseados em métricas de desempenho reflete uma compreensão sofisticada das variações cognitivas presentes no público idoso.

Esta abordagem é particularmente relevante para a educação em segurança da informação, onde conceitos complexos devem ser apresentados de forma acessível e progressiva. A capacidade de ajustar dinamicamente a dificuldade com base no desempenho do usuário permite manter os aprendizes na zona de fluxo adequada, maximizando tanto o engajamento quanto a eficácia do aprendizado.

A integração de metodologias blended learning, combinando elementos analógicos e digitais, representa uma estratégia inovadora para facilitar a transição tecnológica. Esta abordagem reconhece que muitos idosos podem se sentir mais confortáveis com formatos tradicionais, utilizando-os como ponte para o aprendizado digital. No contexto da cibersegurança, esta estratégia pode ser particularmente eficaz para introduzir conceitos abstratos através de analogias concretas e familiares.

O uso de cenários cotidianos familiares, como cozinha, direção e compras, demonstra uma compreensão profunda da importância da relevância contextual no aprendizado de adultos. Esta abordagem não apenas aumenta o engajamento, mas também facilita a transferência de conhecimentos para situações reais, um aspecto crítico para a educação em segurança da informação.

A implementação de Life Story Work, utilizando a história pessoal do usuário como base para o aprendizado, representa uma abordagem particularmente inovadora. Esta metodologia não apenas personaliza a experiência de aprendizado, mas também aproveita a rica experiência de vida dos idosos como contexto para novos conhecimentos, particularmente valioso para discussões sobre segurança digital e proteção de informações pessoais.

A diversificação de tipos de jogos (até 9 tipos diferentes identificados em um único sistema) reflete uma compreensão da necessidade de variedade para manter o engajamento a longo prazo. Esta diversidade também permite abordar diferentes estilos de aprendizagem e preferências individuais, aumentando a eficácia geral do sistema educacional.

\subsection{Motivação e Engajamento}
\label{subsec:motivacao_expandida}

A manutenção da motivação e do engajamento representa um desafio particular no desenvolvimento de jogos para idosos. A análise expandida identificou 19 critérios específicos nesta categoria, conforme apresentado na Tabela \ref{tab:motivacao_expandida}.

\begin{table}[H]
\centering
\caption{Critérios expandidos de Motivação e Engajamento}
\label{tab:motivacao_expandida}
\begin{tabular}{p{2.5cm}p{3.5cm}p{4.5cm}p{2.5cm}}
\hline
\textbf{Subcategoria} & \textbf{Critério} & \textbf{Justificativa/Contexto} & \textbf{Referência} \\ \hline
Prevenção Frustração & Evitar em níveis altos & Manter engajamento em desafios complexos & Zuo et al. (2024) \\
Feedback Adaptativo & IA personalizada & Responder necessidades emocionais & Zuo et al. (2024) \\
Experiência & Desafios progressivos & Evitar monotonia e manter interesse & Zuo et al. (2024) \\
Cooperação & 2-4 jogadores & Promover interação social e suporte & Blažič (2024) \\
Ajuda & Múltiplas formas & Manter engajamento menos habilidosos & Blažič (2024) \\
Recompensas & Pontos, badges, níveis & Fornecer feedback positivo & Blažič (2024) \\
Cartas Joker & Evitar perguntas difíceis & Reduzir ansiedade e manter participação & Blažič (2024) \\
Controle Dificuldade & Sistema adaptativo & Garantir conclusão em qualquer caso & Yang et al. (2024) \\
Dicas & Após 5 falhas & Mostrar dicas após múltiplas tentativas & Yang et al. (2024) \\
Pular Jogo & Após 10 falhas & Permitir pular após muitas tentativas & Yang et al. (2024) \\
Progressão Auto & Após inatividade & Avançar automaticamente após 10s & Yang et al. (2024) \\
Feedback Dinâmico & Sobre ações & Fornecer mensagens sobre correção & Yang et al. (2024) \\
Orientação & Passo-a-passo & Dar orientação no início de rodadas & Yang et al. (2024) \\
Quiz 4 Opções & 3 falsas, 1 verdadeira & Implementar formato quiz padrão & Baptista et al. (2022) \\
Pontuação & Baseada tentativas & Calcular pontuação por número tentativas & Rummun \& Nagowah (2022) \\
Cronômetros & Para dificuldade & Usar timers para aumentar foco & Rummun \& Nagowah (2022) \\
Multiplayer & Interação social & Incluir jogos para dois jogadores & Rummun \& Nagowah (2022) \\
Superar Medos & Erros online & Ajudar superar medo de erros & Bernardino et al. (2021) \\
Wow Effect & AR visual & Usar efeito visual AR para engajamento & Boletsis \& McCallum (2023) \\
\hline
\end{tabular}
\end{table}

Os critérios de motivação e engajamento revelam uma compreensão profunda das necessidades psicológicas do público idoso. A prevenção de frustração emerge como um tema central, reconhecendo que experiências negativas podem levar ao abandono da atividade. Esta preocupação é particularmente relevante no contexto da educação em segurança da informação, onde a complexidade dos conceitos pode facilmente gerar ansiedade.

O sistema de controle de dificuldade adaptativo, que garante a conclusão do jogo em qualquer caso, representa uma abordagem inovadora para manter o engajamento. A implementação de dicas após 5 tentativas falhadas e a opção de pular após 10 tentativas demonstra uma compreensão sofisticada do equilíbrio entre desafio e frustração. A progressão automática após 10 segundos de inatividade garante que usuários não fiquem "presos" em situações difíceis.

A implementação de sistemas cooperativos (2-4 jogadores) representa uma estratégia valiosa para aproveitar a natureza social do aprendizado. Para idosos, que podem enfrentar isolamento social, os jogos cooperativos oferecem não apenas benefícios educacionais, mas também oportunidades de interação social significativa. No contexto da cibersegurança, esta abordagem pode facilitar a discussão e o compartilhamento de experiências relacionadas a ameaças digitais.

O uso de inteligência artificial para feedback adaptativo personalizado representa uma fronteira emergente no campo. Esta tecnologia permite que sistemas respondam adequadamente às necessidades emocionais individuais, adaptando não apenas o conteúdo, mas também o tom e a frequência do feedback para maximizar a motivação.

A implementação de "cartas joker" para evitar perguntas difíceis ocasionalmente demonstra uma compreensão sensível da necessidade de reduzir ansiedade e manter participação. Esta mecânica permite que usuários mantenham o senso de controle sobre sua experiência de aprendizado, um aspecto crítico para a autoeficácia.

O "wow effect" da realidade aumentada representa uma estratégia tecnológica para capturar o engajamento inicial. A visualização de artefatos 3D sobre a visão do mundo real pode provocar fascínio e curiosidade, elementos essenciais para motivar a exploração continuada do sistema.

\subsection{Segurança e Privacidade}
\label{subsec:seguranca_expandida}

A categoria de segurança e privacidade, significativamente expandida durante esta análise, representa uma das contribuições mais importantes desta investigação. Foram identificados 16 critérios específicos, conforme apresentado na Tabela \ref{tab:seguranca_expandida}.

\begin{table}[H]
\centering
\caption{Critérios expandidos de Segurança e Privacidade}
\label{tab:seguranca_expandida}
\begin{tabular}{p{2.5cm}p{3.5cm}p{4.5cm}p{2.5cm}}
\hline
\textbf{Subcategoria} & \textbf{Critério} & \textbf{Justificativa/Contexto} & \textbf{Referência} \\ \hline
Coleta Dados & Avaliações cognitivas & Monitoramento médico sem interrupção & Zuo et al. (2024) \\
Monitoramento & EEG e ECG & Garantir bem-estar durante experiência & Zuo et al. (2024) \\
Proteção & Dados responsável & Proteger privacidade informações sensíveis & Zuo et al. (2024) \\
Educação Digital & Segurança internet & Preparar para uso seguro tecnologias & Blažič (2024) \\
Módulos Segurança & 4 específicos & Privacidade, Compras, Spams, Vírus & Machado et al. (2017) \\
Situações Reais & Vivências cotidianas & Usar situações baseadas experiências reais & Machado et al. (2017) \\
Pensamento Crítico & Uso seguro internet & Desenvolver criticidade para segurança & Machado et al. (2017) \\
Reflexão & Segurança privacidade & Possibilitar questionamento sobre segurança & Machado et al. (2017) \\
Ameaças Específicas & Spam, vírus, phishing & Abordar ameaças específicas & Machado et al. (2017) \\
Tarefas Cotidianas & Online interconectadas & Redes sociais, e-mails, compras online & Bernardino et al. (2021) \\
Perigos Online & Destacar riscos & Evidenciar perigos atividades online & Bernardino et al. (2021) \\
Comportamento Seguro & Escolhas adequadas & Ensinar atitude segura diante ameaças & Bernardino et al. (2021) \\
Consciência Ameaças & Cibersegurança & Aumentar consciência sobre ameaças & Bernardino et al. (2021) \\
Medidas Preventivas & Ações segurança & Ensinar medidas preventivas & Bernardino et al. (2021) \\
Vulnerabilidades & Específicas idosos & Estudar vulnerabilidades do público & Bernardino et al. (2021) \\
Usuários Seguros & Web consciente & Tornar usuários conscientes e seguros & Bernardino et al. (2021) \\
\hline
\end{tabular}
\end{table}

Os critérios de segurança e privacidade refletem uma preocupação dual: proteger os usuários durante o uso do jogo e educá-los sobre segurança digital. Esta dualidade é particularmente importante para o público idoso, que frequentemente representa um grupo vulnerável tanto em termos de proteção de dados quanto de conhecimento sobre ameaças digitais.

A implementação de quatro módulos específicos de segurança (Privacidade, Compras online, Spams e Vírus) representa uma abordagem estruturada para abordar as principais ameaças enfrentadas por idosos no ambiente digital. Esta categorização permite um tratamento sistemático de diferentes tipos de riscos, facilitando tanto o aprendizado quanto a aplicação prática dos conhecimentos adquiridos.

O desenvolvimento do pensamento crítico para uso seguro da internet emerge como um objetivo fundamental, transcendendo o simples ensino de regras específicas. Esta abordagem reconhece que as ameaças digitais evoluem constantemente, tornando essencial desenvolver capacidades de avaliação e tomada de decisão que possam ser aplicadas a novas situações.

A integração de avaliações cognitivas e monitoramento fisiológico (EEG e ECG) levanta questões importantes sobre o equilíbrio entre benefícios terapêuticos e privacidade. Estes critérios sugerem a necessidade de frameworks éticos robustos para orientar o desenvolvimento de jogos sérios que coletam dados sensíveis de usuários vulneráveis.

A interconexão de tarefas cotidianas online (redes sociais, e-mails, compras online) reflete uma compreensão realista do ambiente digital moderno. Esta abordagem holística reconhece que a segurança digital não pode ser tratada de forma isolada, mas deve ser integrada a todas as atividades online dos usuários.

O foco em vulnerabilidades específicas dos idosos demonstra uma compreensão das características particulares deste grupo demográfico que os tornam alvos preferenciais para fraudes digitais. Esta especialização é essencial para desenvolver estratégias de proteção eficazes e relevantes.

\subsection{Acessibilidade e Inclusão}
\label{subsec:acessibilidade_expandida}

A categoria de acessibilidade e inclusão aborda as necessidades específicas do público idoso em termos de limitações físicas e cognitivas. Foram identificados 18 critérios específicos, conforme apresentado na Tabela \ref{tab:acessibilidade_expandida}.

\begin{table}[H]
\centering
\caption{Critérios expandidos de Acessibilidade e Inclusão}
\label{tab:acessibilidade_expandida}
\begin{tabular}{p{2.5cm}p{3.5cm}p{4.5cm}p{2.5cm}}
\hline
\textbf{Subcategoria} & \textbf{Critério} & \textbf{Justificativa/Contexto} & \textbf{Referência} \\ \hline
Ambiente & Confortável & Diminuir ansiedade tecnológica & Blažič (2024) \\
Familiaridade & Conceitos conhecidos & Facilitar adoção por público menos familiarizado & Blažič (2024) \\
Suporte & Colaborativo & Garantir inclusão todos níveis habilidade & Blažič (2024) \\
Tecnologia & Touchscreen tablets & Aproveitar tecnologia promissora & Blažič (2024) \\
Compatibilidade & Adaptive Performance & Compatibilidade diferentes dispositivos Android & Yang et al. (2024) \\
Deficiências & Suporte específico & Considerar usuários com deficiências & Yang et al. (2024) \\
Vida Independente & Foco pacientes sozinhos & Considerar pacientes que vivem sozinhos & Yang et al. (2024) \\
Simplicidade & Processo direto & Garantir processo jogo simples & Yang et al. (2024) \\
Áudio & Botão reprodução & Permitir ouvir todo texto da tela & Baptista et al. (2022) \\
Leitura & Assistência & Ajudar com dificuldades limitações leitura & Baptista et al. (2022) \\
Letramento & Baixo nível & Considerar usuários baixa alfabetização & Baptista et al. (2022) \\
Supervisão & Inicial opcional & Permitir uso com assistente saúde & Baptista et al. (2022) \\
Lares Idosos & Adaptação & Adaptar para uso casas repouso & Rummun \& Nagowah (2022) \\
Habilidades & Diferentes níveis & Acomodar usuários diferentes capacidades & Rummun \& Nagowah (2022) \\
Leitores Tela & Recurso específico & Implementar todo conteúdo formato texto & Machado et al. (2017) \\
Equipe & Interdisciplinar & Envolver múltiplas áreas conhecimento & Machado et al. (2017) \\
Padrões W3C & Acessibilidade & Seguir recomendações acessibilidade & Machado et al. (2017) \\
Ambientes Domésticos & Uso casa & Adaptar para uso configurações domésticas & Boletsis \& McCallum (2023) \\
\hline
\end{tabular}
\end{table}

Os critérios de acessibilidade e inclusão enfatizam a importância de criar ambientes de aprendizado que acomodem as diversas necessidades do público idoso. A criação de atmosferas confortáveis e o uso de conceitos familiares representam estratégias fundamentais para reduzir barreiras de entrada tecnológica.

A implementação de recursos para leitores de tela, que disponibilizam todo o conteúdo em formato texto, demonstra um compromisso com a acessibilidade universal. Esta funcionalidade é particularmente importante para usuários com limitações visuais ou dificuldades de leitura, garantindo que o conteúdo educacional seja acessível independentemente das capacidades individuais.

O suporte colaborativo emerge como um mecanismo importante para garantir que jogadores com diferentes níveis de habilidade possam participar efetivamente. Esta abordagem é particularmente valiosa no contexto da educação em segurança da informação, onde a diversidade de conhecimentos prévios pode ser significativa.

A consideração de usuários com baixo letramento reflete uma compreensão realista da diversidade educacional presente no público idoso. Esta sensibilidade é essencial para garantir que soluções tecnológicas não excluam inadvertidamente usuários com diferentes backgrounds educacionais.

A adaptação para diferentes ambientes de uso (casas de repouso, configurações domésticas) demonstra flexibilidade no design de sistemas. Esta versatilidade é importante para maximizar o alcance e a utilidade das soluções desenvolvidas.

O envolvimento de equipes interdisciplinares (pedagogia, design, gerontologia, letras, assistência social) reflete a complexidade inerente ao desenvolvimento de soluções para este público. Esta abordagem multidisciplinar é essencial para garantir que todos os aspectos das necessidades dos usuários sejam adequadamente considerados.

\subsection{Avaliação e Coleta de Dados}
\label{subsec:avaliacao_expandida}

A categoria de avaliação e coleta de dados aborda os métodos e técnicas para medir a eficácia dos jogos sérios e coletar informações sobre o desempenho dos usuários. Foram identificados 18 critérios específicos, conforme apresentado na Tabela \ref{tab:avaliacao_expandida}.

\begin{table}[H]
\centering
\caption{Critérios expandidos de Avaliação e Coleta de Dados}
\label{tab:avaliacao_expandida}
\begin{tabular}{p{2.5cm}p{3.5cm}p{4.5cm}p{2.5cm}}
\hline
\textbf{Subcategoria} & \textbf{Critério} & \textbf{Justificativa/Contexto} & \textbf{Referência} \\ \hline
Armazenamento & Automático respostas & Salvar automaticamente banco dados & Baptista et al. (2022) \\
Progresso & Indicação atividades & Mostrar quais perguntas respondidas & Baptista et al. (2022) \\
Adição & Novas respostas & Permitir adicionar respostas não presentes & Baptista et al. (2022) \\
Edição & Correção respostas & Possibilitar correção respostas incorretas & Baptista et al. (2022) \\
Tempo Reação & Medição testes & Medir tempo reação testes velocidade & Rummun \& Nagowah (2022) \\
Tentativas & Contagem & Registrar número tentativas para pontuação & Rummun \& Nagowah (2022) \\
Respostas Tempo & Corretas período & Contar respostas corretas 30 segundos & Rummun \& Nagowah (2022) \\
Sequências & Memorização & Avaliar capacidade memorizar reproduzir & Rummun \& Nagowah (2022) \\
Avaliação & Qualitativa quantitativa & Usar dados qualitativos e quantitativos & Machado et al. (2017) \\
Questionário & Pós-uso & Aplicar questionário após utilização & Machado et al. (2017) \\
Observação & Participante & Realizar observação durante uso & Machado et al. (2017) \\
Análise Conteúdo & Metodologia Moraes & Usar análise conteúdo específica & Machado et al. (2017) \\
Validação & Com usuários & Validar com grupo idosos & Machado et al. (2017) \\
Experience Quest & In-Game & Usar questionário experiência jogo & Boletsis \& McCallum (2023) \\
Usability Scale & Sistema & Aplicar escala usabilidade sistema & Boletsis \& McCallum (2023) \\
Entrevistas & Semi-estruturadas & Conduzir entrevistas abertas & Boletsis \& McCallum (2023) \\
Medições & Precisão erro & Avaliar tarefas usando medidas precisão & Boletsis \& McCallum (2023) \\
Observações & Específicas usuários & Documentar observações comentários & Boletsis \& McCallum (2023) \\
\hline
\end{tabular}
\end{table}

Os critérios de avaliação e coleta de dados revelam uma tendência crescente para a implementação de sistemas de avaliação integrados, onde métricas comportamentais são coletadas automaticamente durante o gameplay. Esta abordagem permite uma avaliação mais natural e menos intrusiva do desempenho dos usuários, evitando a artificialidade de testes separados.

A utilização de questionários padronizados, como o In-Game Experience Questionnaire e o System Usability Scale, demonstra uma maturidade metodológica crescente no campo. Estes instrumentos validados permitem comparações entre estudos e contribuem para a construção de uma base de conhecimento mais robusta.

A combinação de métodos qualitativos e quantitativos reflete uma compreensão da complexidade inerente à avaliação de sistemas educacionais. Enquanto métricas quantitativas fornecem dados objetivos sobre desempenho, métodos qualitativos como entrevistas semi-estruturadas e observação participante oferecem insights sobre a experiência subjetiva dos usuários.

A implementação de sistemas de armazenamento automático de respostas e indicação de progresso demonstra uma preocupação com a experiência do usuário durante o processo de avaliação. Estas funcionalidades reduzem a carga cognitiva associada ao acompanhamento do próprio progresso, permitindo que usuários se concentrem no conteúdo educacional.

A medição de tempo de reação e contagem de tentativas fornece dados objetivos sobre aspectos cognitivos específicos, particularmente relevantes para avaliações de declínio cognitivo ou eficácia de intervenções. Estas métricas podem ser especialmente valiosas no contexto de jogos terapêuticos para idosos.

A validação com grupos de usuários reais representa um aspecto crítico para garantir a relevância e eficácia das soluções desenvolvidas. Esta prática garante que os sistemas atendam às necessidades reais dos usuários, não apenas às expectativas dos desenvolvedores.


\section{Questões de Pesquisa}
\label{sec:qp_expandida}

A realização da análise expandida da Revisão Sistemática da Literatura (RSL) possibilitou respostas abrangentes e fundamentadas às questões de pesquisa propostas. A identificação de 120 critérios e recomendações específicos, organizados em seis categorias principais, fornece subsídios quantitativos e qualitativos robustos para embasar as respostas às questões centrais desta investigação.

\subsection{Quais as recomendações, requisitos, e critérios identificados no desenvolvimento de jogos para o público idoso?}
\label{subsec:qp1_expandida}

A análise expandida dos artigos selecionados permitiu a identificação de 120 critérios e recomendações específicos, organizados em seis categorias principais. A distribuição quantitativa revela que Aspectos Pedagógicos e de Aprendizagem (26 critérios, 21.7\%), Design de Interface e Usabilidade (23 critérios, 19.2\%), e Motivação e Engajamento (19 critérios, 15.8\%) representam as áreas de maior concentração de recomendações, seguidas por Acessibilidade e Inclusão (18 critérios, 15.0\%), Avaliação e Coleta de Dados (18 critérios, 15.0\%), e Segurança e Privacidade (16 critérios, 13.3\%).

Esta distribuição equilibrada reflete uma abordagem holística que reconhece a interconexão entre aspectos técnicos, educacionais e psicológicos no desenvolvimento de jogos eficazes. A categoria de Aspectos Pedagógicos e de Aprendizagem, com a maior concentração de critérios, evidencia a importância fundamental da base educacional sólida para o sucesso de jogos sérios.

Dentre as recomendações mais frequentemente mencionadas destacam-se: o uso de progressão gradual de dificuldade com múltiplos níveis (tipicamente 5 níveis por fase, progredindo através de 8 níveis crescentes), a implementação de sistemas de adaptação dinâmica baseados em métricas de desempenho, e a utilização de cenários cotidianos familiares para aumentar relevância e engajamento.

A análise revelou também a importância crítica de critérios relacionados à prevenção de frustração e manutenção do engajamento. A implementação de sistemas de controle de dificuldade adaptativos, que garantem a conclusão do jogo em qualquer caso, representa uma inovação significativa para manter usuários motivados mesmo diante de desafios complexos.

Particularmente relevante é a identificação de critérios específicos para tecnologias emergentes, como realidade aumentada, que oferece oportunidades únicas para criar experiências de aprendizado mais imersivas e intuitivas. A capacidade de projetar conteúdo sobre objetos físicos representa uma fronteira promissora para o desenvolvimento futuro de jogos sérios para idosos.

A categoria de Design de Interface e Usabilidade revelou critérios específicos baseados em evidências empíricas, como a preferência por plataformas Android (55\% de popularidade vs 23\% iOS entre idosos) e a utilização de gráficos 2D para reduzir requisitos de hardware e evitar tonturas. Estas recomendações demonstram a importância de decisões de design baseadas em dados concretos sobre o público-alvo.

\subsection{Como jogos desenvolvidos tendo o público idoso como alvo são validados e testados?}
\label{subsec:qp2_expandida}

A análise expandida revelou uma evolução significativa nas metodologias de validação e teste de jogos para idosos. Foram identificados 18 critérios específicos relacionados à avaliação e coleta de dados, demonstrando uma crescente sofisticação metodológica no campo.

A validação de jogos voltados ao público idoso ocorre através de múltiplas abordagens complementares. A tendência mais significativa é a implementação de sistemas de avaliação integrados, onde métricas comportamentais são coletadas automaticamente durante o gameplay. Esta abordagem permite uma avaliação mais natural e menos intrusiva, evitando a artificialidade de testes separados.

A utilização de questionários padronizados representa uma maturidade metodológica crescente. Instrumentos como o In-Game Experience Questionnaire, System Usability Scale, e metodologias específicas como a análise de conteúdo proposta por Moraes, permitem comparações entre estudos e contribuem para a construção de uma base de conhecimento mais robusta.

A combinação de métodos qualitativos e quantitativos emergiu como uma prática padrão. Enquanto métricas quantitativas (tempo de reação, número de tentativas, respostas corretas por período) fornecem dados objetivos sobre desempenho, métodos qualitativos como entrevistas semi-estruturadas e observação participante oferecem insights sobre a experiência subjetiva dos usuários.

A validação através de frameworks padronizados, como o DigComp 2.2 (Framework Europeu de Competências Digitais), representa uma tendência importante para garantir alinhamento com padrões reconhecidos internacionalmente. Esta abordagem facilita a comparabilidade entre estudos e a replicabilidade de resultados.

Uma inovação significativa identificada é a implementação de Game-Based Assessment (GBA), que combina entretenimento com avaliação efetiva. Esta metodologia permite que a avaliação ocorra de forma integrada ao gameplay, reduzindo a ansiedade associada a testes formais e fornecendo dados mais representativos do desempenho real dos usuários.

A validação com grupos de usuários reais, envolvendo tipicamente 26 ou mais participantes idosos, representa um aspecto crítico para garantir a relevância e eficácia das soluções desenvolvidas. Esta prática garante que os sistemas atendam às necessidades reais dos usuários, não apenas às expectativas dos desenvolvedores.

\subsection{Que abordagens são usadas para levantamento de requisitos nestes jogos e quais produzem melhores resultados?}
\label{subsec:qp3_expandida}

A análise expandida revelou uma diversificação significativa nas abordagens para levantamento de requisitos, com a emergência de metodologias híbridas que combinam múltiplas técnicas para maximizar a eficácia do processo de desenvolvimento.

Foram identificadas cinco abordagens principais para levantamento de requisitos: revisão sistemática da literatura seguida de validação empírica, design participativo com envolvimento direto do público-alvo, consulta a especialistas multidisciplinares, metodologia Action Research, e implementação de equipes interdisciplinares.

A metodologia Action Research emergiu como uma abordagem particularmente eficaz, focando na identificação de problemas específicos e desenvolvimento de soluções através de ciclos iterativos de planejamento, ação, observação e análise. Esta abordagem permite uma resposta mais ágil às necessidades identificadas durante o processo de desenvolvimento.

A implementação de equipes interdisciplinares, envolvendo áreas como pedagogia, design, gerontologia, letras, assistência social, e ciência da computação, mostrou-se fundamental para garantir que todos os aspectos das necessidades dos usuários sejam adequadamente considerados. Esta diversidade de perspectivas é essencial para o desenvolvimento de soluções verdadeiramente inclusivas e eficazes.

A metodologia ConstruMED para Construção de Materiais Educacionais Digitais, baseada em design pedagógico, representa uma abordagem estruturada específica para o desenvolvimento de jogos educacionais. Esta metodologia, composta por cinco etapas (Preparação, Implementação, Avaliação e Distribuição), fornece um framework sistemático para o desenvolvimento de soluções educacionais.

A análise revelou que abordagens híbridas, combinando revisão da literatura com validação empírica e design participativo, produzem resultados mais robustos. A implementação de metodologias blended, que combinam elementos analógicos e digitais, emerge como uma estratégia particularmente eficaz para facilitar a transição tecnológica do público idoso.

Uma descoberta significativa é a importância do envolvimento contínuo dos usuários finais durante todo o processo de desenvolvimento, não apenas nas fases de validação final. Esta abordagem participativa garante que as soluções desenvolvidas atendam às necessidades reais dos usuários e sejam culturalmente apropriadas.

\section{Desenvolvimento de Jogos Relacionados à Cibersegurança}
\label{sec:ciberseguranca_expandida}

A análise expandida revelou uma lacuna crítica na literatura atual: apenas 2 dos 8 artigos analisados em profundidade abordam especificamente aspectos de cibersegurança para idosos. Esta baixa representatividade evidencia uma necessidade urgente de pesquisas nesta área, especialmente considerando a crescente vulnerabilidade deste público a ameaças digitais.

Os estudos que cumpriram critérios relacionados à cibersegurança (Machado et al., 2017; Bernardino et al., 2021) forneceram contribuições valiosas para a compreensão de como jogos sérios podem ser utilizados para educação em segurança da informação. Estes trabalhos permitiram a identificação de 16 critérios específicos para a categoria de Segurança e Privacidade, representando uma expansão significativa em relação ao conhecimento previamente disponível.

O estudo de Machado et al. (2017) apresenta o objeto de aprendizagem "SegurIdade Virtual", estruturado em quatro módulos específicos: Privacidade, Comprando online, Spams e Vírus online. Esta abordagem modular permite um tratamento sistemático de diferentes tipos de ameaças, facilitando tanto o aprendizado quanto a aplicação prática dos conhecimentos adquiridos.

O desenvolvimento do pensamento crítico para uso seguro da internet emerge como um objetivo fundamental, transcendendo o simples ensino de regras específicas. Esta abordagem reconhece que as ameaças digitais evoluem constantemente, tornando essencial desenvolver capacidades de avaliação e tomada de decisão que possam ser aplicadas a novas situações.

O estudo de Bernardino et al. (2021) propõe uma abordagem baseada em narrativas que interconectam tarefas cotidianas dos idosos (navegação web, redes sociais, e-mails, compras online), destacando os perigos que estas atividades apresentam e ensinando como escolher comportamentos seguros diante de ameaças iminentes.

\subsection{Critérios Específicos para Cibersegurança}
\label{subsec:criterios_ciberseguranca}

A análise dos estudos específicos sobre cibersegurança revelou critérios únicos que não aparecem em outras categorias de jogos sérios para idosos. Estes critérios podem ser organizados em quatro subcategorias principais:

\textbf{Educação em Ameaças Específicas:} A implementação de módulos dedicados a ameaças específicas (spam, vírus, phishing, fraudes em compras online) permite um tratamento detalhado de cada tipo de risco. Esta abordagem especializada é essencial dado que diferentes ameaças requerem diferentes estratégias de proteção.

\textbf{Desenvolvimento de Pensamento Crítico:} O foco no desenvolvimento de capacidades de avaliação crítica, ao invés de simplesmente ensinar regras específicas, prepara os usuários para enfrentar ameaças emergentes. Esta abordagem é particularmente importante dado o ritmo acelerado de evolução das ameaças digitais.

\textbf{Contextualização em Atividades Cotidianas:} A integração de conceitos de segurança em atividades online familiares (redes sociais, e-mails, compras) aumenta a relevância e facilita a transferência de conhecimentos para situações reais.

\textbf{Reflexão sobre Vulnerabilidades Pessoais:} O estímulo à reflexão sobre vulnerabilidades específicas dos idosos e suas implicações para a segurança digital promove uma compreensão mais profunda dos riscos pessoais.

\subsection{Implicações para Pesquisas Futuras}
\label{subsec:implicacoes_futuras}

A escassez de estudos específicos sobre jogos de cibersegurança para idosos, combinada com os critérios gerais identificados, sugere direções importantes para pesquisas futuras. A integração dos 120 critérios identificados pode informar o desenvolvimento de jogos sérios mais eficazes para educação em segurança da informação.

A categoria de Design de Interface e Usabilidade ganha dimensão especial na educação em cibersegurança, onde a identificação de ameaças frequentemente depende de sinais visuais sutis. A adaptação destes critérios para cenários de segurança pode incluir a implementação de sistemas de alerta visuais intuitivos e a utilização de metáforas familiares para conceitos de segurança complexos.

Os Aspectos Pedagógicos e de Aprendizagem assumem importância crítica na educação em cibersegurança, onde a progressão gradual de complexidade pode começar com conceitos básicos de proteção de dados pessoais e evoluir para tópicos mais sofisticados como reconhecimento de engenharia social e gestão de identidade digital.

A categoria de Motivação e Engajamento torna-se crítica no contexto da cibersegurança, onde tópicos podem ser percebidos como intimidantes ou irrelevantes. A implementação de sistemas cooperativos pode facilitar a discussão de experiências pessoais com ameaças digitais, criando oportunidades valiosas para aprendizado peer-to-peer e desmistificação de conceitos de segurança.

\section{Síntese dos Resultados e Contribuições}
\label{sec:sintese_expandida}

A análise sistemática expandida dos critérios e recomendações de design para jogos sérios voltados ao público idoso revelou um campo de pesquisa em rápida evolução, caracterizado por uma crescente sofisticação metodológica e uma compreensão aprofundada das necessidades específicas deste público. A identificação de 120 critérios organizados em seis categorias principais fornece um framework abrangente para orientar o desenvolvimento futuro de soluções educacionais.

\subsection{Contribuições Teóricas}
\label{subsec:contribuicoes_teoricas}

A taxonomia expandida desenvolvida nesta investigação representa uma contribuição teórica significativa para o campo, fornecendo a primeira categorização sistemática de critérios específicos para jogos sérios de cibersegurança para idosos. A estrutura proposta integra conhecimentos de múltiplas disciplinas, oferecendo um framework organizacional que facilita tanto a pesquisa acadêmica quanto a aplicação prática.

A identificação de padrões emergentes, como a importância da adaptação tecnológica gradual, personalização multidimensional, e integração de aspectos sociais, fornece insights valiosos para o desenvolvimento de teorias mais robustas sobre aprendizado digital no envelhecimento. Estes padrões transcendem as categorias individuais, sugerindo princípios fundamentais que devem orientar o design de tecnologias educacionais para idosos.

A expansão significativa da categoria de Segurança e Privacidade, de 4 para 16 critérios específicos, estabelece uma base teórica importante para pesquisas futuras nesta área crítica. A crescente importância da educação em cibersegurança para populações vulneráveis torna esta contribuição particularmente relevante para políticas públicas e práticas de desenvolvimento.

\subsection{Contribuições Metodológicas}
\label{subsec:contribuicoes_metodologicas}

A metodologia de análise desenvolvida, combinando análise temática com categorização sistemática baseada em frameworks estabelecidos, oferece um modelo replicável para investigações similares. A integração de múltiplas fontes de evidência e a validação através de trabalhos de referência aumenta a robustez e confiabilidade dos resultados.

A identificação de métodos de validação emergentes, como Game-Based Assessment e sistemas de avaliação integrados, contribui para o avanço das práticas metodológicas no campo. Estas abordagens oferecem alternativas mais naturais e menos intrusivas para a avaliação de jogos educacionais.

A documentação detalhada de abordagens para levantamento de requisitos, particularmente metodologias híbridas e equipes interdisciplinares, fornece orientação prática para pesquisadores e desenvolvedores interessados em criar soluções similares.

\subsection{Contribuições Práticas}
\label{subsec:contribuicoes_praticas}

O framework de 120 critérios organizados em seis categorias oferece orientação prática imediata para desenvolvedores de jogos sérios. A estrutura detalhada, com justificativas e contextos específicos, facilita a aplicação prática dos critérios identificados.

A identificação de tecnologias específicas (Android como plataforma preferencial, gráficos 2D, realidade aumentada) e especificações técnicas (dimensões de tabuleiros, número de níveis de dificuldade) fornece orientação concreta para decisões de implementação.

A documentação de critérios específicos para cibersegurança oferece um ponto de partida valioso para o desenvolvimento de jogos educacionais nesta área crítica, preenchendo uma lacuna significativa na literatura disponível.

\subsection{Limitações e Direções Futuras}
\label{subsec:limitacoes_direções}

Embora esta investigação tenha analisado 8 artigos representativos dos 62 selecionados, a análise de todos os artigos poderia revelar critérios adicionais e padrões mais refinados. A continuação desta análise expandida representa uma oportunidade importante para enriquecer ainda mais o framework desenvolvido.

A concentração geográfica e cultural dos estudos analisados, com predominância de pesquisas conduzidas em contextos europeus e norte-americanos, sugere a necessidade de expansão para contextos culturais mais diversos. Esta diversificação poderia revelar critérios e necessidades específicas que não foram capturadas na literatura atual.

A escassez de estudos longitudinais sobre a eficácia a longo prazo de jogos sérios para idosos representa uma lacuna importante que requer atenção futura. O desenvolvimento de metodologias para avaliação longitudinal poderia contribuir significativamente para a compreensão da sustentabilidade e impacto duradouro destas intervenções.

A integração de tecnologias emergentes, como inteligência artificial para personalização adaptativa e realidade virtual para experiências imersivas, representa fronteiras promissoras que requerem investigação adicional. O desenvolvimento de critérios específicos para estas tecnologias poderia expandir significativamente as possibilidades de design.

\subsection{Impacto Esperado}
\label{subsec:impacto_esperado}

A taxonomia desenvolvida tem potencial para influenciar tanto a pesquisa acadêmica quanto a prática profissional no desenvolvimento de jogos sérios para idosos. A disponibilização de um framework sistemático e abrangente pode acelerar o desenvolvimento de soluções mais eficazes e inclusivas.

A identificação de lacunas críticas, particularmente na área de cibersegurança, pode estimular o desenvolvimento de novas linhas de pesquisa e atrair financiamento para investigações nesta área prioritária. A crescente digitalização da sociedade torna esta contribuição cada vez mais relevante.

A metodologia desenvolvida pode ser adaptada para outros contextos e populações, contribuindo para o desenvolvimento de frameworks mais amplos para design de tecnologia educacional inclusiva. Esta transferibilidade aumenta o impacto potencial da investigação além do escopo específico dos jogos para idosos.

Em conclusão, os resultados apresentados fornecem uma base sólida para o avanço do conhecimento em jogos sérios para idosos, particularmente no contexto emergente da educação em cibersegurança. As direções futuras identificadas oferecem um roadmap claro para pesquisas subsequentes, enquanto os critérios e padrões identificados podem orientar o desenvolvimento prático de soluções mais eficazes e inclusivas. A contribuição desta investigação estende-se além do campo específico dos jogos sérios, oferecendo insights valiosos para o design de tecnologias educacionais que atendam às necessidades de populações vulneráveis em um mundo cada vez mais digitalizado.



\section{Tabelas Complementares de Critérios Expandidos}
\label{sec:tabelas_complementares}

Com base na análise expandida dos artigos selecionados, apresentam-se a seguir as tabelas complementares que detalham os critérios identificados nas categorias de Segurança e Privacidade, Acessibilidade e Inclusão, e Avaliação e Coleta de Dados. Estas tabelas representam uma contribuição significativa para a compreensão dos requisitos específicos para o desenvolvimento de jogos sérios voltados ao público idoso.

\begin{table}[H]
\centering
\caption{Critérios de Segurança e Privacidade para Jogos Sérios voltados ao Público Idoso}
\label{tab:seguranca_privacidade}
\begin{tabular}{p{0.5cm}p{3cm}p{4cm}p{4cm}p{2.5cm}}
\hline
\textbf{ID} & \textbf{Critério} & \textbf{Descrição} & \textbf{Justificativa/Contexto} & \textbf{Referência} \\ \hline
SP01 & Coleta responsável de dados cognitivos & Implementar sistemas de monitoramento médico sem interrupção da experiência de jogo & Permitir avaliações cognitivas contínuas mantendo o engajamento do usuário & Zuo et al. (2024) \\
SP02 & Monitoramento fisiológico seguro & Integrar sensores EEG e ECG com protocolos de segurança rigorosos & Garantir bem-estar do usuário durante experiências imersivas & Zuo et al. (2024) \\
SP03 & Proteção de dados sensíveis & Implementar frameworks robustos para proteção de informações pessoais e médicas & Proteger privacidade de informações sensíveis coletadas durante o jogo & Zuo et al. (2024) \\
SP04 & Educação em segurança digital & Integrar conteúdos sobre uso seguro de tecnologias e internet & Preparar idosos para uso seguro e consciente de tecnologias digitais & Blažič (2024) \\
SP05 & Módulos específicos de segurança & Implementar 4 módulos: Privacidade, Compras online, Spams e Vírus & Abordar sistematicamente as principais ameaças digitais enfrentadas por idosos & Machado et al. (2017) \\
SP06 & Situações baseadas em vivências reais & Usar cenários de segurança baseados em experiências cotidianas dos idosos & Aumentar relevância e facilitar transferência de conhecimentos para situações reais & Machado et al. (2017) \\
SP07 & Desenvolvimento de pensamento crítico & Promover capacidade de avaliação crítica para uso seguro da internet & Desenvolver habilidades de análise que transcendem regras específicas & Machado et al. (2017) \\
SP08 & Reflexão sobre segurança pessoal & Possibilitar questionamento e reflexão sobre práticas de segurança e privacidade & Estimular consciência sobre vulnerabilidades pessoais e suas implicações & Machado et al. (2017) \\
SP09 & Educação sobre ameaças específicas & Abordar ameaças como spam, vírus, phishing e fraudes online & Fornecer conhecimento específico sobre tipos de ameaças mais comuns & Machado et al. (2017) \\
SP10 & Interconexão de tarefas online & Integrar cenários de redes sociais, e-mails e compras online & Refletir a realidade interconectada do ambiente digital moderno & Bernardino et al. (2021) \\
SP11 & Destaque de perigos online & Evidenciar riscos e perigos presentes em atividades digitais cotidianas & Aumentar consciência sobre ameaças presentes em atividades aparentemente seguras & Bernardino et al. (2021) \\
SP12 & Ensino de comportamentos seguros & Instruir sobre escolha de atitudes adequadas diante de ameaças digitais & Capacitar usuários para tomada de decisões seguras em tempo real & Bernardino et al. (2021) \\
SP13 & Consciência sobre cibersegurança & Aumentar conhecimento geral sobre ameaças e vulnerabilidades digitais & Desenvolver compreensão abrangente do panorama de ameaças digitais & Bernardino et al. (2021) \\
SP14 & Medidas preventivas integradas & Ensinar ações preventivas e medidas de segurança proativa & Capacitar usuários para prevenção ao invés de apenas reação a ameaças & Bernardino et al. (2021) \\
SP15 & Foco em vulnerabilidades específicas & Estudar e abordar vulnerabilidades particulares do público idoso & Reconhecer características que tornam idosos alvos preferenciais para fraudes & Bernardino et al. (2021) \\
SP16 & Formação de usuários conscientes & Transformar usuários em navegadores conscientes e seguros da web & Objetivo final de autonomia e segurança digital para o público idoso & Bernardino et al. (2021) \\
\hline
\end{tabular}
\end{table}


\begin{table}[H]
\centering
\caption{Critérios de Acessibilidade e Inclusão para Jogos Sérios voltados ao Público Idoso}
\label{tab:acessibilidade_inclusao}
\begin{tabular}{p{0.5cm}p{3cm}p{4cm}p{4cm}p{2.5cm}}
\hline
\textbf{ID} & \textbf{Critério} & \textbf{Descrição} & \textbf{Justificativa/Contexto} & \textbf{Referência} \\ \hline
AI01 & Ambiente confortável & Criar atmosfera acolhedora e não intimidante para interação tecnológica & Diminuir ansiedade tecnológica e facilitar adoção por usuários menos familiarizados & Blažič (2024) \\
AI02 & Uso de conceitos familiares & Implementar elementos e metáforas conhecidas pelo público idoso & Facilitar compreensão e reduzir barreiras de entrada tecnológica & Blažič (2024) \\
AI03 & Suporte colaborativo & Garantir mecanismos de ajuda e suporte entre usuários & Assegurar inclusão de usuários com diferentes níveis de habilidade tecnológica & Blažič (2024) \\
AI04 & Tecnologia touchscreen otimizada & Aproveitar potencial de tablets com interface de toque & Utilizar tecnologia promissora e intuitiva para interação direta & Blažič (2024) \\
AI05 & Compatibilidade multiplataforma & Implementar Adaptive Performance para diferentes dispositivos Android & Garantir funcionamento adequado em dispositivos com capacidades variadas & Yang et al. (2024) \\
AI06 & Suporte para deficiências & Considerar usuários com limitações visuais, auditivas ou motoras & Garantir acessibilidade universal independentemente de limitações específicas & Yang et al. (2024) \\
AI07 & Foco em vida independente & Considerar necessidades de pacientes que vivem sozinhos & Desenvolver soluções que funcionem sem supervisão constante & Yang et al. (2024) \\
AI08 & Simplicidade de processo & Garantir que o processo de jogo seja direto e intuitivo & Reduzir complexidade cognitiva e facilitar uso independente & Yang et al. (2024) \\
AI09 & Funcionalidade de áudio & Implementar botão para reprodução de todo texto da tela & Permitir acesso auditivo ao conteúdo para usuários com dificuldades visuais & Baptista et al. (2022) \\
AI10 & Assistência para leitura & Fornecer suporte para usuários com dificuldades ou limitações de leitura & Garantir acesso ao conteúdo independentemente de habilidades de leitura & Baptista et al. (2022) \\
AI11 & Adaptação para baixo letramento & Considerar usuários com baixo nível de alfabetização & Garantir inclusão de usuários com diferentes backgrounds educacionais & Baptista et al. (2022) \\
AI12 & Supervisão opcional & Permitir uso com assistente de saúde quando necessário & Oferecer flexibilidade para diferentes níveis de autonomia & Baptista et al. (2022) \\
AI13 & Adaptação para lares de idosos & Customizar interface e funcionalidades para uso em casas de repouso & Adequar solução para diferentes ambientes de cuidado & Rummun \& Nagowah (2022) \\
AI14 & Acomodação de habilidades diversas & Adaptar jogos para usuários com diferentes capacidades cognitivas e físicas & Reconhecer heterogeneidade do público idoso em termos de capacidades & Rummun \& Nagowah (2022) \\
AI15 & Compatibilidade com leitores de tela & Implementar todo conteúdo em formato acessível para tecnologias assistivas & Garantir acessibilidade para usuários com deficiências visuais & Machado et al. (2017) \\
AI16 & Desenvolvimento interdisciplinar & Envolver equipe com pedagogia, design, gerontologia, letras e assistência social & Garantir abordagem holística que considere todos aspectos das necessidades dos usuários & Machado et al. (2017) \\
AI17 & Conformidade com padrões W3C & Seguir recomendações internacionais de acessibilidade web & Garantir aderência a padrões reconhecidos de acessibilidade digital & Machado et al. (2017) \\
AI18 & Uso em ambientes domésticos & Adaptar para uso em configurações domésticas e familiares & Permitir utilização em ambiente natural e confortável do usuário & Boletsis \& McCallum (2023) \\
\hline
\end{tabular}
\end{table}


\begin{table}[H]
\centering
\caption{Critérios de Avaliação e Coleta de Dados para Jogos Sérios voltados ao Público Idoso}
\label{tab:avaliacao_coleta_dados}
\begin{tabular}{p{0.5cm}p{3cm}p{4cm}p{4cm}p{2.5cm}}
\hline
\textbf{ID} & \textbf{Critério} & \textbf{Descrição} & \textbf{Justificativa/Contexto} & \textbf{Referência} \\ \hline
ACD01 & Armazenamento automático & Salvar automaticamente todas as respostas em banco de dados & Reduzir carga cognitiva do usuário e garantir integridade dos dados coletados & Baptista et al. (2022) \\
ACD02 & Indicação de progresso & Mostrar claramente quais perguntas ou atividades foram respondidas & Fornecer feedback visual sobre avanço e permitir retomada de atividades & Baptista et al. (2022) \\
ACD03 & Adição de novas respostas & Permitir que usuários adicionem respostas não presentes nas opções & Capturar diversidade de experiências e perspectivas não antecipadas & Baptista et al. (2022) \\
ACD04 & Edição de respostas & Possibilitar correção de respostas incorretas ou inadequadas & Permitir refinamento de dados e reduzir ansiedade sobre erros & Baptista et al. (2022) \\
ACD05 & Medição de tempo de reação & Registrar tempo de resposta em testes de velocidade cognitiva & Obter métricas objetivas sobre capacidades cognitivas específicas & Rummun \& Nagowah (2022) \\
ACD06 & Contagem de tentativas & Registrar número de tentativas necessárias para completar tarefas & Avaliar dificuldade percebida e adaptar sistema de pontuação & Rummun \& Nagowah (2022) \\
ACD07 & Respostas corretas por período & Contar respostas corretas em janelas temporais específicas (ex: 30 segundos) & Medir velocidade de processamento e eficiência cognitiva & Rummun \& Nagowah (2022) \\
ACD08 & Avaliação de sequências & Medir capacidade de memorizar e reproduzir sequências & Avaliar memória de trabalho e capacidades de sequenciamento & Rummun \& Nagowah (2022) \\
ACD09 & Metodologia mista & Combinar dados qualitativos e quantitativos na avaliação & Obter compreensão abrangente tanto de desempenho quanto de experiência & Machado et al. (2017) \\
ACD10 & Questionário pós-uso & Aplicar questionário estruturado após utilização do jogo & Capturar percepções e experiências subjetivas dos usuários & Machado et al. (2017) \\
ACD11 & Observação participante & Realizar observação sistemática durante o uso do sistema & Identificar comportamentos e dificuldades não capturados por outros métodos & Machado et al. (2017) \\
ACD12 & Análise de conteúdo estruturada & Usar metodologia específica (Moraes) para análise qualitativa & Garantir rigor metodológico na interpretação de dados qualitativos & Machado et al. (2017) \\
ACD13 & Validação com usuários reais & Conduzir testes com grupos representativos de idosos & Garantir relevância e aplicabilidade dos resultados para o público-alvo & Machado et al. (2017) \\
ACD14 & Questionário de experiência no jogo & Aplicar In-Game Experience Questionnaire padronizado & Usar instrumento validado para medir qualidade da experiência de jogo & Boletsis \& McCallum (2023) \\
ACD15 & Escala de usabilidade & Implementar System Usability Scale para avaliação sistemática & Obter métricas padronizadas de usabilidade comparáveis entre estudos & Boletsis \& McCallum (2023) \\
ACD16 & Entrevistas semi-estruturadas & Conduzir entrevistas abertas para aprofundamento qualitativo & Capturar insights detalhados sobre experiência e percepções dos usuários & Boletsis \& McCallum (2023) \\
ACD17 & Medições de precisão e erro & Avaliar tarefas usando métricas objetivas de acurácia & Quantificar desempenho em tarefas específicas com precisão & Boletsis \& McCallum (2023) \\
ACD18 & Documentação de observações específicas & Registrar sistematicamente observações e comentários dos usuários & Capturar detalhes contextuais importantes para interpretação dos resultados & Boletsis \& McCallum (2023) \\
\hline
\end{tabular}
\end{table}


\subsection{Análise das Tabelas Complementares}
\label{subsec:analise_tabelas_complementares}

As três tabelas complementares apresentadas (Tabelas \ref{tab:seguranca_privacidade}, \ref{tab:acessibilidade_inclusao} e \ref{tab:avaliacao_coleta_dados}) representam uma contribuição significativa para a compreensão dos requisitos específicos para o desenvolvimento de jogos sérios voltados ao público idoso, particularmente no contexto da educação em cibersegurança e letramento digital.

\subsubsection{Critérios de Segurança e Privacidade}
\label{subsubsec:analise_seguranca}

A Tabela \ref{tab:seguranca_privacidade} revela uma abordagem dual para segurança em jogos sérios: proteção dos usuários durante o uso do sistema e educação sobre segurança digital. Esta dualidade é particularmente importante para o público idoso, que frequentemente representa um grupo vulnerável tanto em termos de proteção de dados quanto de conhecimento sobre ameaças digitais.

Os critérios SP05 a SP09, derivados do trabalho de Machado et al. (2017), estabelecem uma estrutura modular para educação em cibersegurança, abordando sistematicamente as principais ameaças: privacidade, compras online, spam e vírus. Esta abordagem estruturada permite um tratamento detalhado de cada tipo de risco, facilitando tanto o aprendizado quanto a aplicação prática dos conhecimentos adquiridos.

Os critérios SP10 a SP16, baseados no estudo de Bernardino et al. (2021), enfatizam a interconexão das atividades digitais modernas e a necessidade de desenvolver consciência crítica sobre segurança. O foco no desenvolvimento de pensamento crítico (SP07) transcende o simples ensino de regras específicas, preparando os usuários para enfrentar ameaças emergentes.

\subsubsection{Critérios de Acessibilidade e Inclusão}
\label{subsubsec:analise_acessibilidade}

A Tabela \ref{tab:acessibilidade_inclusao} demonstra uma compreensão abrangente das diversas necessidades do público idoso. Os critérios identificados abordam desde limitações físicas específicas (AI06, AI09, AI15) até considerações socioculturais mais amplas (AI11, AI16).

Particularmente relevante é o critério AI16, que enfatiza a necessidade de equipes interdisciplinares envolvendo pedagogia, design, gerontologia, letras e assistência social. Esta abordagem multidisciplinar reconhece a complexidade inerente ao desenvolvimento de soluções para este público e a necessidade de expertise diversificada.

Os critérios AI01 a AI04 refletem uma preocupação fundamental com a redução de barreiras psicológicas à adoção tecnológica. A criação de ambientes confortáveis e o uso de conceitos familiares são estratégias essenciais para superar a ansiedade tecnológica comum entre idosos.

\subsubsection{Critérios de Avaliação e Coleta de Dados}
\label{subsubsec:analise_avaliacao}

A Tabela \ref{tab:avaliacao_coleta_dados} revela uma evolução significativa nas metodologias de avaliação de jogos sérios. A tendência mais notável é a implementação de sistemas de avaliação integrados (ACD01, ACD02), onde métricas comportamentais são coletadas automaticamente durante o gameplay, evitando a artificialidade de testes separados.

Os critérios ACD14 e ACD15 demonstram uma crescente maturidade metodológica através do uso de instrumentos padronizados como o In-Game Experience Questionnaire e o System Usability Scale. Estes instrumentos validados permitem comparações entre estudos e contribuem para a construção de uma base de conhecimento mais robusta.

A combinação de métodos qualitativos e quantitativos (ACD09, ACD11, ACD16) reflete uma compreensão da complexidade inerente à avaliação de sistemas educacionais. Enquanto métricas quantitativas fornecem dados objetivos sobre desempenho, métodos qualitativos oferecem insights sobre a experiência subjetiva dos usuários.

\subsection{Implicações para o Desenvolvimento de Jogos de Cibersegurança}
\label{subsec:implicacoes_desenvolvimento}

A análise integrada das três tabelas complementares oferece insights valiosos para o desenvolvimento de jogos sérios específicos para educação em cibersegurança para idosos. A categoria de Segurança e Privacidade, significativamente expandida nesta investigação, fornece diretrizes específicas que podem ser integradas com os critérios de outras categorias para criar soluções mais eficazes.

A integração dos critérios de acessibilidade com os de segurança sugere a necessidade de interfaces que não apenas sejam inclusivas, mas também comuniquem efetivamente conceitos de segurança. Por exemplo, o uso de metáforas familiares (AI02) pode ser aplicado para explicar conceitos complexos de cibersegurança através de analogias com situações cotidianas conhecidas pelos idosos.

Os critérios de avaliação fornecem um framework robusto para medir tanto a eficácia educacional quanto a usabilidade de jogos de cibersegurança. A implementação de sistemas de coleta automática de dados (ACD01) combinada com questionários padronizados (ACD14, ACD15) permite uma avaliação abrangente que captura tanto métricas objetivas de aprendizado quanto percepções subjetivas de experiência.

A escassez de estudos específicos sobre jogos de cibersegurança para idosos, evidenciada pela identificação de apenas dois trabalhos relevantes entre os analisados, confirma a importância e originalidade desta linha de pesquisa. Os 52 critérios identificados nestas três categorias complementares fornecem uma base sólida para orientar o desenvolvimento futuro de soluções nesta área crítica.

