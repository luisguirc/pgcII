\chapter{Resultados e Discussões}
\label{cap:resultados}

Ao término do processo de seleção, foram identificados 62 artigos elegíveis para a etapa de extração de recomendações e análise, conforme demonstrado na Tabela \ref{tab:selecao_artigos}. Esses estudos representam um panorama relevante de critérios elencados ou utilizados como referência no desenvolvimento de jogos digitais voltados ao público idoso, incluindo menções pontuais a aspectos relacionados à cibersegurança.

\begin{table}[H]
\centering
\caption{Número de artigos selecionados por fase da RSL}
\label{tab:selecao_artigos}
\begin{tabular}{lcccc|c}
\hline
\textbf{Fase} & \textbf{ACM} & \textbf{Web of Science} & \textbf{IEEE Xplore} & \textbf{SBC Open Lib} & \textbf{Total} \\ \hline
Fase 1        & 14           & 275                     & 167                  & 2                    & 458            \\
Fase 2        & 14           & 251                     & 156                  & 1                    & 422            \\
Fase 3        & 4            & 71                      & 36                   & 1                    & 112            \\
Fase 4        & 1            & 38                      & 22                   & 1                    & 62               \\ \hline
\end{tabular}
\end{table}

A listagem de cada um dos artigos e os critérios analisados em cada fase pode ser acessada em:
\begin{itemize}
    \item Fase 1: \href{http://bit.ly/3GtHj5X}{http://bit.ly/3GtHj5X}
    \item Fase 2: \href{https://bit.ly/4iYA4k1}{https://bit.ly/4iYA4k1}
    \item Fase 3: \href{https://bit.ly/3RIhUrp}{https://bit.ly/3RIhUrp}
    \item Fase 4: \href{https://bit.ly/4m14C7p}{https://bit.ly/4m14C7p}
    \item Resultado: \href{https://bit.ly/3YVzJqN}{https://bit.ly/3YVzJqN}
\end{itemize}

\section{Questões de Pesquisa}\label{sec:qp}

A realização dessas etapas da Revisão Sistemática da Literatura (RSL) possibilitou o início da resposta às questões de pesquisa propostas. A análise detalhada e a subsequente extração de dados dos artigos selecionados, a serem conduzidas nas próximas fases do Projeto de Graduação em Computação (PGC), fornecerão subsídios quantitativos para embasar respostas mais precisas e fundamentadas.

\subsection{Quais as recomendações, requisitos, e critérios identificados no desenvolvimento de jogos para o público idoso?}\label{subsec:qp1}

A maior parte dos estudos analisados apresenta o uso de jogos sérios com foco em objetivos como treinamento cognitivo e prevenção de doenças neurodegenerativas associadas à demência \cite{yang2024serious, zuo2024development, caggianese2018towards}, e promoção do envelhecimento ativo \cite{nacimiento-garcia2024gamification}.

Dentre as recomendações mais frequentemente mencionadas destacam-se: o uso de fontes ampliadas para facilitar a leitura \cite{tziraki2017designing}, a simplificação das interfaces visuais \cite{valladares2017design}, e a possibilidade de personalização de parâmetros como a velocidade do jogo, quando aplicável.

\subsection{Como jogos desenvolvidos tendo o público idoso como alvo são validados e testados?}\label{subsec:qp2}

Com base no levantamento inicial, observou-se que a validação de jogos voltados ao público idoso ocorre majoritariamente por meio da aplicação direta do jogo seguida de coleta de \textit{feedback} dos usuários \cite{merilampi2017cognitive}, pela aplicação de avaliações antes e após a intervenção \cite{wong2022effectiveness}, ou ainda por meio de consultas a especialistas das áreas da saúde e design, como fisioterapeutas, cuidadores e designers \cite{busca2024serious}.

\subsection{Que abordagens são usadas para levantamento de requisitos nestes jogos e quais produzem melhores resultados?}\label{subsec:qp3}

Verificou-se que poucos trabalhos descrevem de forma explícita a metodologia empregada para o levantamento de requisitos. Essa lacuna é abordada de forma específica em \cite{machado2018heuristics, manser2021making}.

Em \cite{machado2018heuristics}, foi conduzida uma revisão da literatura para levantamento de requisitos e princípios de design. Em seguida, jogos sérios para idosos oriundos da revisão da literatura foram testados com o público alvo, e avaliados por meio de um questionário, a fim de entender a validade de tais requisitos. Posteriormente, com base em dados empíricos, o estudo buscou desenvolver uma teoria que descreva características desejáveis no desenvolvimento de tais jogos.

Em \cite{manser2021making}, os autores executaram as seguintes fases: (1) Revisão da literatura. (2) Modelagem do perfil de usuário, definindo aspectos demográficos, capacidades, características, hobbies, e motivação para jogar. (3) Levantamento de necessidades terapêuticas junto a profissionais. (4) Definição de escopo tecnológico, ou seja, quais aparelhos eletrônicos seriam utilizados para o jogo. (5) Estratégia de manutenção, com o objetivo de possibilitar o uso da solução desenvolvida fora do período de estudo. O desenvolvimento do jogo se deu em um \textit{loop} de implementação dos requisitos seguido de validação com usuários.

\section{Desenvolvimento de jogos relacionados à cibersegurança}\label{sec:ciberseguranca}

Durante a etapa de busca, apenas \textbf{3} artigos atenderam ao critério de inclusão 4, que estabelece a necessidade de associação entre as recomendações identificadas e aspectos de cibersegurança ou proteção digital. Os trabalhos que cumpriram esse critério foram \cite{machado2017learning, bernardino2021serious, van2020serious}.

Nestes estudos, destacam-se aspectos como o estímulo ao pensamento crítico sobre segurança na internet por meio de jogos adaptados ao público idoso \cite{machado2017learning}, o uso de mecânicas lúdicas e narrativas cotidianas para promover a alfabetização digital em cibersegurança \cite{bernardino2021serious}, e a aplicação de agentes virtuais com análise vocal para treinar resiliência verbal contra fraudes praticadas presencialmente \cite{van2020serious}. Essas abordagens refletem diferentes estratégias para integrar conteúdos de proteção digital ao design de jogos sérios, com ênfase em situações reais enfrentadas por usuários vulneráveis.

A baixa quantidade de estudos que tratam simultaneamente de recomendações para jogos digitais e de princípios de cibersegurança evidencia uma lacuna significativa na literatura atual. Este cenário aponta para um campo de pesquisa ainda em estágio inicial, com amplo potencial para investigações futuras voltadas à integração entre segurança digital e design de jogos, especialmente para públicos vulneráveis, como o idoso.
