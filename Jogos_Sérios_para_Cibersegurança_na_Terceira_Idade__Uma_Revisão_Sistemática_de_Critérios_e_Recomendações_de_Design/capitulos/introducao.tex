% ----------------------------------------------------------
% Introdução 
% Capítulo sem numeração, mas presente no Sumário
% ----------------------------------------------------------

\chapter[Introdução]{Introdução}
\label{cap:introducao}
% \addcontentsline{toc}{chapter}{Introdução}



 

% Demonstração de citação: o software de análise foi desenvolvido em Python~\cite{van1995python}, utilizando as bibliotecas Pandas~\cite{mckinney2010data} e Scikit-learn~\cite{scikit-learn}.

O envelhecimento da população é um fenômeno global que traz consigo novas demandas sociais e tecnológicas. Com a crescente digitalização de serviços e interações, a população idosa tem se tornado cada vez mais conectada, mas também mais vulnerável a ameaças cibernéticas~\cite{morrison2021how}. Nesse contexto, os jogos sérios (serious games) emergem como uma ferramenta promissora para educação e treinamento, oferecendo um ambiente interativo e motivador para o desenvolvimento de habilidades ~\cite{connolly2012systematic}. Especificamente na área de cibersegurança, os jogos sérios podem desempenhar um papel crucial na capacitação de idosos para navegar no mundo digital com mais segurança.

Embora os benefícios dos jogos sérios para a população idosa sejam reconhecidos, abrangendo melhorias cognitivas, físicas e sociais ~\cite{konstantinidis2010integration, mccallum2012gamification}, existe uma lacuna significativa no que diz respeito a jogos focados em cibersegurança e projetados especificamente para este público. Muitos jogos existentes não consideram adequadamente as necessidades, preferências e limitações dos usuários idosos, comprometendo sua eficácia e engajamento ~\cite{mol2021desirable}. A aplicação de princípios de Desenho Universal e a atenção à acessibilidade são fundamentais, mas ainda pouco exploradas nesse domínio específico ~\cite{belarmino2021criterios}.

O desenvolvimento de jogos sérios eficazes para idosos apresenta desafios únicos, especialmente na fase de levantamento de requisitos ~\cite{beristain2021standardizing}. Garantir que o jogo seja não apenas funcional, mas também acessível, intuitivo e motivador para o público-alvo exige uma compreensão profunda de suas características e expectativas. Metodologias de design centradas no usuário (\textit{User-Centered Design} - UCD) são essenciais ~\cite{brox2017user}, porém, a adaptação de técnicas de elicitação de requisitos para considerar as particularidades cognitivas, físicas e a familiaridade com tecnologia dos idosos ainda é um campo que necessita de maior investigação ~\cite{fua2013designing}.

Diante desse cenário, identifica-se a ausência de um conjunto consolidado de diretrizes e critérios, fundamentado em evidências, que possa guiar o desenvolvimento de jogos sérios sobre cibersegurança para a população idosa. Questões fundamentais permanecem em aberto: Quais são as diretrizes de design essenciais para esses jogos? Como eles devem ser validados junto ao público-alvo? E quais abordagens de levantamento de requisitos se mostram mais eficazes para capturar as necessidades desse grupo específico?

Este trabalho tem como objetivo principal preencher essa lacuna por meio de uma Revisão Sistemática da Literatura (RSL). Procura-se identificar, analisar e sintetizar os critérios, diretrizes, boas práticas e desafios relacionados ao desenvolvimento e avaliação de jogos sérios voltados para pessoas idosas, especialmente no contexto de cibersegurança. Como resultado, pretende-se consolidar um conjunto de recomendações que possa servir como um guia prático para desenvolvedores, designers e pesquisadores interessados em criar soluções de jogos mais eficazes e acessíveis para este público.

A metodologia adotada seguirá um protocolo de RSL, envolvendo buscas sistemáticas em bases de dados acadêmicas relevantes (como \textit{IEEE Xplore}, \textit{ACM Digital Library}, \textit{Web of Science} e \textit{SBC Open Lib}), aplicação de critérios de inclusão e exclusão pré-definidos, e análise qualitativa dos estudos selecionados. Espera-se que a contribuição deste trabalho resida na consolidação do conhecimento existente e na oferta de um referencial prático e baseado em evidências para o avanço da área de jogos sérios para a cibersegurança e inclusão digital da população idosa.

\section{Justificativa}
%\section{Motivação}
A vulnerabilidade crescente de pessoas idosas a ameaças cibernéticas, combinada com o potencial dos jogos sérios como ferramenta educacional engajadora, justifica a necessidade de investigar e consolidar recomendações específicas para o desenvolvimento de tais jogos. A falta de diretrizes claras dificulta a criação de soluções eficazes e acessíveis, reforçando a importância desta pesquisa para orientar futuros desenvolvimentos na área.

\section{Objetivos}

\subsection{Objetivo Geral}
Realizar uma Revisão Sistemática da Literatura (RSL) para identificar e analisar os critérios, diretrizes e metodologias de levantamento de requisitos propostos no desenvolvimento de jogos sérios voltados para o público idoso, com foco em aplicações relacionadas à cibersegurança e segurança digital.

\subsection{Objetivos Específicos}
Os objetivos específicos são:
\begin{itemize}
    \item Apresentar o protocolo da Revisão Sistemática da Literatura (RSL) realizada, incluindo as questões de pesquisa, \textit{strings} de busca, bases de dados consultadas, e critérios de inclusão e exclusão.
    \item Descrever o processo de seleção dos artigos relevantes.
    \item Apresentar os resultados preliminares da RSL, com foco na identificação dos artigos selecionados que abordam recomendações para jogos sérios para idosos, destacando aqueles relacionados à cibersegurança.
    \item Discutir a metodologia de trabalho e o cronograma previsto para as próximas fases do projeto (PGC-II e PGC-III), que incluirão a análise aprofundada dos artigos selecionados, e a consolidação dos resultados.
\end{itemize}

%sugestão
\section{Estrutura da Monografia}
O presente relatório está estruturado da seguinte forma: o Capítulo~\ref{cap:fundamentacao_teorica} apresenta a fundamentação teórica sobre jogos sérios, o público idoso e tecnologia, e cibersegurança. O Capítulo~\ref{cap:trabalhos_relacionados} discute trabalhos relacionados, incluindo outras revisões de literatura. O Capítulo~\ref{cap:metodologia} detalha a metodologia da RSL e o cronograma do projeto. O Capítulo~\ref{cap:resultados} apresenta os resultados preliminares da busca e seleção de artigos. Finalmente, o Capítulo~\ref{cap:conclusoes} sumariza os achados desta fase inicial e delineia os próximos passos da pesquisa.

% SUGESTAO: Exemplo de estrutura da monografia:

% O presente relatório está estruturado da seguinte forma: o Capítulo~\ref{cap:fundamentacao_teorica} apresenta..., o Capítulo~\ref{cap:trabalhos_relacionados}... O Capítulo~\ref{cap:metodologia} ..., o Capítulo~\ref{cap:resultados} .... Finalmente, o Capítulo~\ref{cap:conclusao} apresenta ...



