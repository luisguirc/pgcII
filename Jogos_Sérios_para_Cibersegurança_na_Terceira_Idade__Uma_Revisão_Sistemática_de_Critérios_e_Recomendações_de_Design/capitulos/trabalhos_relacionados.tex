\chapter{Trabalhos Relacionados}\label{cap:trabalhos_relacionados}

A relação entre jogos digitais e segurança da informação no contexto da terceira idade tem sido explorada por diferentes estudos, com enfoques variados em alfabetização digital, reabilitação cognitiva, usabilidade e proteção contra golpes virtuais. Esta seção apresenta os principais trabalhos identificados que abordam tais temas, com destaque para suas propostas, metodologias e contribuições.

Um estudo relevante nesse contexto é o de Pillon e Silva \cite{pillon2022checklist}, que propuseram um checklist estruturado com 30 itens organizados nas categorias de terapia, motivação, interação e segurança, com o objetivo de orientar o desenvolvimento e a avaliação de jogos digitais em realidade virtual para a reabilitação de idosos. O checklist foi fundamentado em uma revisão sistemática de literatura e validado por meio de entrevistas semiestruturadas com especialistas, sendo uma importante ferramenta metodológica para garantir a adequação dos jogos às necessidades do público sênior.

O jogo Quiz60+ foi desenvolvido com o objetivo de instruir idosos sobre os principais golpes virtuais, incluindo fraudes via WhatsApp, sites falsos e mensagens armadilhas. Seu conteúdo foi estruturado em formato de quiz com múltipla escolha, oferecendo feedback imediato e interface intuitiva. A validação envolveu um pequeno grupo de participantes que testaram o jogo em dispositivos Android e responderam a um questionário de avaliação \cite{amorim2024quiz}.

Outro trabalho relevante é o de \cite{scarpioni2016desenvolvimento}, que propôs ambientes simulados de e-mail e redes sociais com foco na identificação de riscos durante a navegação. Apesar de não constituírem jogos no sentido tradicional, os simuladores visam promover o aprendizado por meio de interação e reflexão sobre comportamentos digitais inseguros.

Além dos jogos com foco em segurança digital, outros trabalhos investigam o uso de jogos sérios na reabilitação cognitiva e física de idosos. Tais estudos, embora não abordem diretamente a cibersegurança, oferecem contribuições importantes sobre usabilidade, motivação e adaptação de interfaces para esse público.

Um levantamento sistemático de grande abrangência foi realizado por \cite{canapa2025interactive}, que analisaram 113 estudos sobre jogos sérios interativos voltados ao treinamento cognitivo de adultos mais velhos. Os autores examinaram dimensões como as tecnologias interativas utilizadas (com destaque para tablets, VR/AR e computadores), os tipos de atividades propostas (como quebra-cabeças, simuladores de tarefas cotidianas e jogos de memória), os perfis dos usuários (incluindo idosos saudáveis, com comprometimento cognitivo leve ou demência) e os efeitos dessas intervenções sobre funções cognitivas específicas como memória, atenção e funções executivas. O trabalho destaca a predominância do uso de tablets, a recorrência de atividades baseadas em correspondência de pares e simulações do cotidiano, e o uso ainda incipiente de tecnologias imersivas. Apesar de seu escopo abrangente, o estudo se concentra majoritariamente em aspectos técnicos e clínicos dos jogos sérios para reabilitação cognitiva, não abordando diretamente os desafios relacionados à segurança digital ou à alfabetização em cibersegurança do público idoso.

Dentre os trabalhos relacionados, destacam-se revisões de literatura que apresentam objetivos semelhantes aos deste estudo, embora divirjam em escopo ou foco.

\cite{tong2014cognitive} explora o desenvolvimento de um jogo sério voltado à avaliação cognitiva de idosos, com foco particular na detecção de condições transitórias como o delirium — um desafio clínico de alta relevância, mas frequentemente subdiagnosticado. O trabalho descreve o processo iterativo de prototipação envolvendo diferentes plataformas (\textit{Wiimote}, \textit{Kinect} e \textit{tablets}) até a escolha final por um jogo de acertar alvos em dispositivos com tela sensível ao toque. A escolha baseou-se em sua portabilidade, baixo custo e maior acessibilidade para pacientes idosos. O jogo foi projetado para medir a capacidade de inibição, uma função executiva frequentemente comprometida com o envelhecimento, e passou por testes de usabilidade com adultos e idosos. Os resultados evidenciaram correlações significativas entre o desempenho no jogo e testes cognitivos padronizados, embora tenham sido apontadas limitações ergonômicas da interface para o público idoso.

\cite{ning2020review} apresentam uma revisão abrangente sobre o uso de jogos sérios no cuidado de pessoas com demência, classificando os jogos conforme os sintomas em diferentes estágios da doença e propondo um modelo de avaliação baseado em múltiplos métodos e grupos. Embora o foco em jogos voltados à população idosa seja um ponto de convergência com este trabalho, a abordagem dos autores centra-se na reabilitação cognitiva e física de doenças neurodegenerativas, sem diretamente elencar recomendações. Diferentemente disso, esta pesquisa sistematiza critérios e recomendações de design voltados à criação de jogos sérios que auxiliem a inclusão digital segura de idosos, com ênfase na proteção contra ameaças digitais.

\cite{lee2019study} exploram os fatores de design de jogos sérios voltados à população idosa, com ênfase na interseção entre propósito funcional e elementos de diversão. O trabalho propõe um modelo conceitual para guiar o design de jogos para esse público, destacando a necessidade de superar barreiras culturais e técnicas na Coreia do Sul. Apesar da relevância ao discutir características específicas do público sênior, o artigo não trata de domínios como a cibersegurança, tampouco aborda metodologias de levantamento de requisitos centradas no usuário idoso.

\cite{cedillo2025systematic} realizam uma revisão sistemática sobre tecnologias de interação humano-computador aplicadas à saúde física e cognitiva de idosos, identificando dispositivos de captura de movimento e abordagens multimodais de interação. Embora também utilizem uma abordagem sistemática e abordem o uso de jogos sérios e exergames, seu foco está na tecnologia assistiva para reabilitação e suporte à saúde, não considerando aplicações educacionais no contexto de segurança digital. Em contraste, o presente estudo propõe diretrizes específicas para o design de jogos educativos sobre cibersegurança voltados a esse público, considerando limitações cognitivas e sensoriais decorrentes do envelhecimento.

\cite{postal2019user} oferecem uma revisão sistemática sobre métodos de avaliação de interfaces voltadas a idosos, com foco em aplicações de realidade virtual. Os autores destacam a falta de metodologias consolidadas para avaliação de usabilidade em interfaces tridimensionais destinadas a esse público. Embora compartilhe com este trabalho a preocupação com acessibilidade e adequação de tecnologias interativas ao perfil da população idosa, a proposta de Postal e Rieder está centrada na avaliação de usabilidade, enquanto este estudo foca na definição de critérios de design e conteúdo para jogos sérios.

\cite{bassano2022visualization} revisam tecnologias de visualização e interação utilizadas em jogos sérios e exergames para avaliação e treinamento cognitivo. O estudo destaca o uso crescente de realidade virtual e aumentada e discute a validação das soluções propostas, mas observa limitações na adoção clínica e na disponibilidade de dados.

De modo geral, os trabalhos analisados revelam uma preocupação crescente com o treinamento cognitivo, a usabilidade e o letramento digital de idosos. No entanto, observa-se uma lacuna significativa na sistematização de critérios e recomendações que integrem, de forma estruturada, princípios de segurança da informação ao design de jogos sérios voltados a esse público. Embora alguns estudos abordem indiretamente temas relacionados à proteção digital, poucos exploram a cibersegurança como eixo central no desenvolvimento das soluções. Além disso, é raro encontrar abordagens que combinem metodologias de levantamento participativo de requisitos com diretrizes de acessibilidade e princípios do desenho universal, de modo a contemplar de forma inclusiva as diferentes necessidades cognitivas, motoras e tecnológicas do público idoso.

Este trabalho diferencia-se dos citados anteriormente nos seguintes aspectos: (i) realiza uma análise específica e sistemática da interseção entre segurança digital e o design de jogos sérios voltados ao público idoso — uma abordagem ainda pouco explorada e consolidada na literatura; (ii) propõe a consolidação de um conjunto de critérios e diretrizes organizados em forma de checklist, com o objetivo de orientar o desenvolvimento de jogos sérios que sejam acessíveis, engajadores e alinhados às práticas de proteção digital; e (iii) adota uma abordagem que vai além do uso de jogos sérios como ferramentas exclusivamente voltadas ao treinamento cognitivo ou à prevenção de doenças neurodegenerativas.
